\documentclass[11pt]{article}

    \usepackage[breakable]{tcolorbox}
    \usepackage{parskip} % Stop auto-indenting (to mimic markdown behaviour)
    
    \usepackage{iftex}
    \ifPDFTeX
    	\usepackage[T1]{fontenc}
    	\usepackage{mathpazo}
    \else
    	\usepackage{fontspec}
    \fi

    % Basic figure setup, for now with no caption control since it's done
    % automatically by Pandoc (which extracts ![](path) syntax from Markdown).
    \usepackage{graphicx}
    % Maintain compatibility with old templates. Remove in nbconvert 6.0
    \let\Oldincludegraphics\includegraphics
    % Ensure that by default, figures have no caption (until we provide a
    % proper Figure object with a Caption API and a way to capture that
    % in the conversion process - todo).
    \usepackage{caption}
    \DeclareCaptionFormat{nocaption}{}
    \captionsetup{format=nocaption,aboveskip=0pt,belowskip=0pt}

    \usepackage[Export]{adjustbox} % Used to constrain images to a maximum size
    \adjustboxset{max size={0.9\linewidth}{0.9\paperheight}}
    \usepackage{float}
    \floatplacement{figure}{H} % forces figures to be placed at the correct location
    \usepackage{xcolor} % Allow colors to be defined
    \usepackage{enumerate} % Needed for markdown enumerations to work
    \usepackage{geometry} % Used to adjust the document margins
    \usepackage{amsmath} % Equations
    \usepackage{amssymb} % Equations
    \usepackage{textcomp} % defines textquotesingle
    % Hack from http://tex.stackexchange.com/a/47451/13684:
    \AtBeginDocument{%
        \def\PYZsq{\textquotesingle}% Upright quotes in Pygmentized code
    }
    \usepackage{upquote} % Upright quotes for verbatim code
    \usepackage{eurosym} % defines \euro
    \usepackage[mathletters]{ucs} % Extended unicode (utf-8) support
    \usepackage{fancyvrb} % verbatim replacement that allows latex
    \usepackage{grffile} % extends the file name processing of package graphics 
                         % to support a larger range
    \makeatletter % fix for grffile with XeLaTeX
    \def\Gread@@xetex#1{%
      \IfFileExists{"\Gin@base".bb}%
      {\Gread@eps{\Gin@base.bb}}%
      {\Gread@@xetex@aux#1}%
    }
    \makeatother

    % The hyperref package gives us a pdf with properly built
    % internal navigation ('pdf bookmarks' for the table of contents,
    % internal cross-reference links, web links for URLs, etc.)
    \usepackage{hyperref}
    % The default LaTeX title has an obnoxious amount of whitespace. By default,
    % titling removes some of it. It also provides customization options.
    \usepackage{titling}
    \usepackage{longtable} % longtable support required by pandoc >1.10
    \usepackage{booktabs}  % table support for pandoc > 1.12.2
    \usepackage[inline]{enumitem} % IRkernel/repr support (it uses the enumerate* environment)
    \usepackage[normalem]{ulem} % ulem is needed to support strikethroughs (\sout)
                                % normalem makes italics be italics, not underlines
    \usepackage{mathrsfs}
    

    
    % Colors for the hyperref package
    \definecolor{urlcolor}{rgb}{0,.145,.698}
    \definecolor{linkcolor}{rgb}{.71,0.21,0.01}
    \definecolor{citecolor}{rgb}{.12,.54,.11}

    % ANSI colors
    \definecolor{ansi-black}{HTML}{3E424D}
    \definecolor{ansi-black-intense}{HTML}{282C36}
    \definecolor{ansi-red}{HTML}{E75C58}
    \definecolor{ansi-red-intense}{HTML}{B22B31}
    \definecolor{ansi-green}{HTML}{00A250}
    \definecolor{ansi-green-intense}{HTML}{007427}
    \definecolor{ansi-yellow}{HTML}{DDB62B}
    \definecolor{ansi-yellow-intense}{HTML}{B27D12}
    \definecolor{ansi-blue}{HTML}{208FFB}
    \definecolor{ansi-blue-intense}{HTML}{0065CA}
    \definecolor{ansi-magenta}{HTML}{D160C4}
    \definecolor{ansi-magenta-intense}{HTML}{A03196}
    \definecolor{ansi-cyan}{HTML}{60C6C8}
    \definecolor{ansi-cyan-intense}{HTML}{258F8F}
    \definecolor{ansi-white}{HTML}{C5C1B4}
    \definecolor{ansi-white-intense}{HTML}{A1A6B2}
    \definecolor{ansi-default-inverse-fg}{HTML}{FFFFFF}
    \definecolor{ansi-default-inverse-bg}{HTML}{000000}

    % commands and environments needed by pandoc snippets
    % extracted from the output of `pandoc -s`
    \providecommand{\tightlist}{%
      \setlength{\itemsep}{0pt}\setlength{\parskip}{0pt}}
    \DefineVerbatimEnvironment{Highlighting}{Verbatim}{commandchars=\\\{\}}
    % Add ',fontsize=\small' for more characters per line
    \newenvironment{Shaded}{}{}
    \newcommand{\KeywordTok}[1]{\textcolor[rgb]{0.00,0.44,0.13}{\textbf{{#1}}}}
    \newcommand{\DataTypeTok}[1]{\textcolor[rgb]{0.56,0.13,0.00}{{#1}}}
    \newcommand{\DecValTok}[1]{\textcolor[rgb]{0.25,0.63,0.44}{{#1}}}
    \newcommand{\BaseNTok}[1]{\textcolor[rgb]{0.25,0.63,0.44}{{#1}}}
    \newcommand{\FloatTok}[1]{\textcolor[rgb]{0.25,0.63,0.44}{{#1}}}
    \newcommand{\CharTok}[1]{\textcolor[rgb]{0.25,0.44,0.63}{{#1}}}
    \newcommand{\StringTok}[1]{\textcolor[rgb]{0.25,0.44,0.63}{{#1}}}
    \newcommand{\CommentTok}[1]{\textcolor[rgb]{0.38,0.63,0.69}{\textit{{#1}}}}
    \newcommand{\OtherTok}[1]{\textcolor[rgb]{0.00,0.44,0.13}{{#1}}}
    \newcommand{\AlertTok}[1]{\textcolor[rgb]{1.00,0.00,0.00}{\textbf{{#1}}}}
    \newcommand{\FunctionTok}[1]{\textcolor[rgb]{0.02,0.16,0.49}{{#1}}}
    \newcommand{\RegionMarkerTok}[1]{{#1}}
    \newcommand{\ErrorTok}[1]{\textcolor[rgb]{1.00,0.00,0.00}{\textbf{{#1}}}}
    \newcommand{\NormalTok}[1]{{#1}}
    
    % Additional commands for more recent versions of Pandoc
    \newcommand{\ConstantTok}[1]{\textcolor[rgb]{0.53,0.00,0.00}{{#1}}}
    \newcommand{\SpecialCharTok}[1]{\textcolor[rgb]{0.25,0.44,0.63}{{#1}}}
    \newcommand{\VerbatimStringTok}[1]{\textcolor[rgb]{0.25,0.44,0.63}{{#1}}}
    \newcommand{\SpecialStringTok}[1]{\textcolor[rgb]{0.73,0.40,0.53}{{#1}}}
    \newcommand{\ImportTok}[1]{{#1}}
    \newcommand{\DocumentationTok}[1]{\textcolor[rgb]{0.73,0.13,0.13}{\textit{{#1}}}}
    \newcommand{\AnnotationTok}[1]{\textcolor[rgb]{0.38,0.63,0.69}{\textbf{\textit{{#1}}}}}
    \newcommand{\CommentVarTok}[1]{\textcolor[rgb]{0.38,0.63,0.69}{\textbf{\textit{{#1}}}}}
    \newcommand{\VariableTok}[1]{\textcolor[rgb]{0.10,0.09,0.49}{{#1}}}
    \newcommand{\ControlFlowTok}[1]{\textcolor[rgb]{0.00,0.44,0.13}{\textbf{{#1}}}}
    \newcommand{\OperatorTok}[1]{\textcolor[rgb]{0.40,0.40,0.40}{{#1}}}
    \newcommand{\BuiltInTok}[1]{{#1}}
    \newcommand{\ExtensionTok}[1]{{#1}}
    \newcommand{\PreprocessorTok}[1]{\textcolor[rgb]{0.74,0.48,0.00}{{#1}}}
    \newcommand{\AttributeTok}[1]{\textcolor[rgb]{0.49,0.56,0.16}{{#1}}}
    \newcommand{\InformationTok}[1]{\textcolor[rgb]{0.38,0.63,0.69}{\textbf{\textit{{#1}}}}}
    \newcommand{\WarningTok}[1]{\textcolor[rgb]{0.38,0.63,0.69}{\textbf{\textit{{#1}}}}}
    
    
    % Define a nice break command that doesn't care if a line doesn't already
    % exist.
    \def\br{\hspace*{\fill} \\* }
    % Math Jax compatibility definitions
    \def\gt{>}
    \def\lt{<}
    \let\Oldtex\TeX
    \let\Oldlatex\LaTeX
    \renewcommand{\TeX}{\textrm{\Oldtex}}
    \renewcommand{\LaTeX}{\textrm{\Oldlatex}}
    % Document parameters
    % Document title
    \title{lab-2}
    
    
    
    
    
% Pygments definitions
\makeatletter
\def\PY@reset{\let\PY@it=\relax \let\PY@bf=\relax%
    \let\PY@ul=\relax \let\PY@tc=\relax%
    \let\PY@bc=\relax \let\PY@ff=\relax}
\def\PY@tok#1{\csname PY@tok@#1\endcsname}
\def\PY@toks#1+{\ifx\relax#1\empty\else%
    \PY@tok{#1}\expandafter\PY@toks\fi}
\def\PY@do#1{\PY@bc{\PY@tc{\PY@ul{%
    \PY@it{\PY@bf{\PY@ff{#1}}}}}}}
\def\PY#1#2{\PY@reset\PY@toks#1+\relax+\PY@do{#2}}

\@namedef{PY@tok@w}{\def\PY@tc##1{\textcolor[rgb]{0.73,0.73,0.73}{##1}}}
\@namedef{PY@tok@c}{\let\PY@it=\textit\def\PY@tc##1{\textcolor[rgb]{0.24,0.48,0.48}{##1}}}
\@namedef{PY@tok@cp}{\def\PY@tc##1{\textcolor[rgb]{0.61,0.40,0.00}{##1}}}
\@namedef{PY@tok@k}{\let\PY@bf=\textbf\def\PY@tc##1{\textcolor[rgb]{0.00,0.50,0.00}{##1}}}
\@namedef{PY@tok@kp}{\def\PY@tc##1{\textcolor[rgb]{0.00,0.50,0.00}{##1}}}
\@namedef{PY@tok@kt}{\def\PY@tc##1{\textcolor[rgb]{0.69,0.00,0.25}{##1}}}
\@namedef{PY@tok@o}{\def\PY@tc##1{\textcolor[rgb]{0.40,0.40,0.40}{##1}}}
\@namedef{PY@tok@ow}{\let\PY@bf=\textbf\def\PY@tc##1{\textcolor[rgb]{0.67,0.13,1.00}{##1}}}
\@namedef{PY@tok@nb}{\def\PY@tc##1{\textcolor[rgb]{0.00,0.50,0.00}{##1}}}
\@namedef{PY@tok@nf}{\def\PY@tc##1{\textcolor[rgb]{0.00,0.00,1.00}{##1}}}
\@namedef{PY@tok@nc}{\let\PY@bf=\textbf\def\PY@tc##1{\textcolor[rgb]{0.00,0.00,1.00}{##1}}}
\@namedef{PY@tok@nn}{\let\PY@bf=\textbf\def\PY@tc##1{\textcolor[rgb]{0.00,0.00,1.00}{##1}}}
\@namedef{PY@tok@ne}{\let\PY@bf=\textbf\def\PY@tc##1{\textcolor[rgb]{0.80,0.25,0.22}{##1}}}
\@namedef{PY@tok@nv}{\def\PY@tc##1{\textcolor[rgb]{0.10,0.09,0.49}{##1}}}
\@namedef{PY@tok@no}{\def\PY@tc##1{\textcolor[rgb]{0.53,0.00,0.00}{##1}}}
\@namedef{PY@tok@nl}{\def\PY@tc##1{\textcolor[rgb]{0.46,0.46,0.00}{##1}}}
\@namedef{PY@tok@ni}{\let\PY@bf=\textbf\def\PY@tc##1{\textcolor[rgb]{0.44,0.44,0.44}{##1}}}
\@namedef{PY@tok@na}{\def\PY@tc##1{\textcolor[rgb]{0.41,0.47,0.13}{##1}}}
\@namedef{PY@tok@nt}{\let\PY@bf=\textbf\def\PY@tc##1{\textcolor[rgb]{0.00,0.50,0.00}{##1}}}
\@namedef{PY@tok@nd}{\def\PY@tc##1{\textcolor[rgb]{0.67,0.13,1.00}{##1}}}
\@namedef{PY@tok@s}{\def\PY@tc##1{\textcolor[rgb]{0.73,0.13,0.13}{##1}}}
\@namedef{PY@tok@sd}{\let\PY@it=\textit\def\PY@tc##1{\textcolor[rgb]{0.73,0.13,0.13}{##1}}}
\@namedef{PY@tok@si}{\let\PY@bf=\textbf\def\PY@tc##1{\textcolor[rgb]{0.64,0.35,0.47}{##1}}}
\@namedef{PY@tok@se}{\let\PY@bf=\textbf\def\PY@tc##1{\textcolor[rgb]{0.67,0.36,0.12}{##1}}}
\@namedef{PY@tok@sr}{\def\PY@tc##1{\textcolor[rgb]{0.64,0.35,0.47}{##1}}}
\@namedef{PY@tok@ss}{\def\PY@tc##1{\textcolor[rgb]{0.10,0.09,0.49}{##1}}}
\@namedef{PY@tok@sx}{\def\PY@tc##1{\textcolor[rgb]{0.00,0.50,0.00}{##1}}}
\@namedef{PY@tok@m}{\def\PY@tc##1{\textcolor[rgb]{0.40,0.40,0.40}{##1}}}
\@namedef{PY@tok@gh}{\let\PY@bf=\textbf\def\PY@tc##1{\textcolor[rgb]{0.00,0.00,0.50}{##1}}}
\@namedef{PY@tok@gu}{\let\PY@bf=\textbf\def\PY@tc##1{\textcolor[rgb]{0.50,0.00,0.50}{##1}}}
\@namedef{PY@tok@gd}{\def\PY@tc##1{\textcolor[rgb]{0.63,0.00,0.00}{##1}}}
\@namedef{PY@tok@gi}{\def\PY@tc##1{\textcolor[rgb]{0.00,0.52,0.00}{##1}}}
\@namedef{PY@tok@gr}{\def\PY@tc##1{\textcolor[rgb]{0.89,0.00,0.00}{##1}}}
\@namedef{PY@tok@ge}{\let\PY@it=\textit}
\@namedef{PY@tok@gs}{\let\PY@bf=\textbf}
\@namedef{PY@tok@gp}{\let\PY@bf=\textbf\def\PY@tc##1{\textcolor[rgb]{0.00,0.00,0.50}{##1}}}
\@namedef{PY@tok@go}{\def\PY@tc##1{\textcolor[rgb]{0.44,0.44,0.44}{##1}}}
\@namedef{PY@tok@gt}{\def\PY@tc##1{\textcolor[rgb]{0.00,0.27,0.87}{##1}}}
\@namedef{PY@tok@err}{\def\PY@bc##1{{\setlength{\fboxsep}{\string -\fboxrule}\fcolorbox[rgb]{1.00,0.00,0.00}{1,1,1}{\strut ##1}}}}
\@namedef{PY@tok@kc}{\let\PY@bf=\textbf\def\PY@tc##1{\textcolor[rgb]{0.00,0.50,0.00}{##1}}}
\@namedef{PY@tok@kd}{\let\PY@bf=\textbf\def\PY@tc##1{\textcolor[rgb]{0.00,0.50,0.00}{##1}}}
\@namedef{PY@tok@kn}{\let\PY@bf=\textbf\def\PY@tc##1{\textcolor[rgb]{0.00,0.50,0.00}{##1}}}
\@namedef{PY@tok@kr}{\let\PY@bf=\textbf\def\PY@tc##1{\textcolor[rgb]{0.00,0.50,0.00}{##1}}}
\@namedef{PY@tok@bp}{\def\PY@tc##1{\textcolor[rgb]{0.00,0.50,0.00}{##1}}}
\@namedef{PY@tok@fm}{\def\PY@tc##1{\textcolor[rgb]{0.00,0.00,1.00}{##1}}}
\@namedef{PY@tok@vc}{\def\PY@tc##1{\textcolor[rgb]{0.10,0.09,0.49}{##1}}}
\@namedef{PY@tok@vg}{\def\PY@tc##1{\textcolor[rgb]{0.10,0.09,0.49}{##1}}}
\@namedef{PY@tok@vi}{\def\PY@tc##1{\textcolor[rgb]{0.10,0.09,0.49}{##1}}}
\@namedef{PY@tok@vm}{\def\PY@tc##1{\textcolor[rgb]{0.10,0.09,0.49}{##1}}}
\@namedef{PY@tok@sa}{\def\PY@tc##1{\textcolor[rgb]{0.73,0.13,0.13}{##1}}}
\@namedef{PY@tok@sb}{\def\PY@tc##1{\textcolor[rgb]{0.73,0.13,0.13}{##1}}}
\@namedef{PY@tok@sc}{\def\PY@tc##1{\textcolor[rgb]{0.73,0.13,0.13}{##1}}}
\@namedef{PY@tok@dl}{\def\PY@tc##1{\textcolor[rgb]{0.73,0.13,0.13}{##1}}}
\@namedef{PY@tok@s2}{\def\PY@tc##1{\textcolor[rgb]{0.73,0.13,0.13}{##1}}}
\@namedef{PY@tok@sh}{\def\PY@tc##1{\textcolor[rgb]{0.73,0.13,0.13}{##1}}}
\@namedef{PY@tok@s1}{\def\PY@tc##1{\textcolor[rgb]{0.73,0.13,0.13}{##1}}}
\@namedef{PY@tok@mb}{\def\PY@tc##1{\textcolor[rgb]{0.40,0.40,0.40}{##1}}}
\@namedef{PY@tok@mf}{\def\PY@tc##1{\textcolor[rgb]{0.40,0.40,0.40}{##1}}}
\@namedef{PY@tok@mh}{\def\PY@tc##1{\textcolor[rgb]{0.40,0.40,0.40}{##1}}}
\@namedef{PY@tok@mi}{\def\PY@tc##1{\textcolor[rgb]{0.40,0.40,0.40}{##1}}}
\@namedef{PY@tok@il}{\def\PY@tc##1{\textcolor[rgb]{0.40,0.40,0.40}{##1}}}
\@namedef{PY@tok@mo}{\def\PY@tc##1{\textcolor[rgb]{0.40,0.40,0.40}{##1}}}
\@namedef{PY@tok@ch}{\let\PY@it=\textit\def\PY@tc##1{\textcolor[rgb]{0.24,0.48,0.48}{##1}}}
\@namedef{PY@tok@cm}{\let\PY@it=\textit\def\PY@tc##1{\textcolor[rgb]{0.24,0.48,0.48}{##1}}}
\@namedef{PY@tok@cpf}{\let\PY@it=\textit\def\PY@tc##1{\textcolor[rgb]{0.24,0.48,0.48}{##1}}}
\@namedef{PY@tok@c1}{\let\PY@it=\textit\def\PY@tc##1{\textcolor[rgb]{0.24,0.48,0.48}{##1}}}
\@namedef{PY@tok@cs}{\let\PY@it=\textit\def\PY@tc##1{\textcolor[rgb]{0.24,0.48,0.48}{##1}}}

\def\PYZbs{\char`\\}
\def\PYZus{\char`\_}
\def\PYZob{\char`\{}
\def\PYZcb{\char`\}}
\def\PYZca{\char`\^}
\def\PYZam{\char`\&}
\def\PYZlt{\char`\<}
\def\PYZgt{\char`\>}
\def\PYZsh{\char`\#}
\def\PYZpc{\char`\%}
\def\PYZdl{\char`\$}
\def\PYZhy{\char`\-}
\def\PYZsq{\char`\'}
\def\PYZdq{\char`\"}
\def\PYZti{\char`\~}
% for compatibility with earlier versions
\def\PYZat{@}
\def\PYZlb{[}
\def\PYZrb{]}
\makeatother


    % For linebreaks inside Verbatim environment from package fancyvrb. 
    \makeatletter
        \newbox\Wrappedcontinuationbox 
        \newbox\Wrappedvisiblespacebox 
        \newcommand*\Wrappedvisiblespace {\textcolor{red}{\textvisiblespace}} 
        \newcommand*\Wrappedcontinuationsymbol {\textcolor{red}{\llap{\tiny$\m@th\hookrightarrow$}}} 
        \newcommand*\Wrappedcontinuationindent {3ex } 
        \newcommand*\Wrappedafterbreak {\kern\Wrappedcontinuationindent\copy\Wrappedcontinuationbox} 
        % Take advantage of the already applied Pygments mark-up to insert 
        % potential linebreaks for TeX processing. 
        %        {, <, #, %, $, ' and ": go to next line. 
        %        _, }, ^, &, >, - and ~: stay at end of broken line. 
        % Use of \textquotesingle for straight quote. 
        \newcommand*\Wrappedbreaksatspecials {% 
            \def\PYGZus{\discretionary{\char`\_}{\Wrappedafterbreak}{\char`\_}}% 
            \def\PYGZob{\discretionary{}{\Wrappedafterbreak\char`\{}{\char`\{}}% 
            \def\PYGZcb{\discretionary{\char`\}}{\Wrappedafterbreak}{\char`\}}}% 
            \def\PYGZca{\discretionary{\char`\^}{\Wrappedafterbreak}{\char`\^}}% 
            \def\PYGZam{\discretionary{\char`\&}{\Wrappedafterbreak}{\char`\&}}% 
            \def\PYGZlt{\discretionary{}{\Wrappedafterbreak\char`\<}{\char`\<}}% 
            \def\PYGZgt{\discretionary{\char`\>}{\Wrappedafterbreak}{\char`\>}}% 
            \def\PYGZsh{\discretionary{}{\Wrappedafterbreak\char`\#}{\char`\#}}% 
            \def\PYGZpc{\discretionary{}{\Wrappedafterbreak\char`\%}{\char`\%}}% 
            \def\PYGZdl{\discretionary{}{\Wrappedafterbreak\char`\$}{\char`\$}}% 
            \def\PYGZhy{\discretionary{\char`\-}{\Wrappedafterbreak}{\char`\-}}% 
            \def\PYGZsq{\discretionary{}{\Wrappedafterbreak\textquotesingle}{\textquotesingle}}% 
            \def\PYGZdq{\discretionary{}{\Wrappedafterbreak\char`\"}{\char`\"}}% 
            \def\PYGZti{\discretionary{\char`\~}{\Wrappedafterbreak}{\char`\~}}% 
        } 
        % Some characters . , ; ? ! / are not pygmentized. 
        % This macro makes them "active" and they will insert potential linebreaks 
        \newcommand*\Wrappedbreaksatpunct {% 
            \lccode`\~`\.\lowercase{\def~}{\discretionary{\hbox{\char`\.}}{\Wrappedafterbreak}{\hbox{\char`\.}}}% 
            \lccode`\~`\,\lowercase{\def~}{\discretionary{\hbox{\char`\,}}{\Wrappedafterbreak}{\hbox{\char`\,}}}% 
            \lccode`\~`\;\lowercase{\def~}{\discretionary{\hbox{\char`\;}}{\Wrappedafterbreak}{\hbox{\char`\;}}}% 
            \lccode`\~`\:\lowercase{\def~}{\discretionary{\hbox{\char`\:}}{\Wrappedafterbreak}{\hbox{\char`\:}}}% 
            \lccode`\~`\?\lowercase{\def~}{\discretionary{\hbox{\char`\?}}{\Wrappedafterbreak}{\hbox{\char`\?}}}% 
            \lccode`\~`\!\lowercase{\def~}{\discretionary{\hbox{\char`\!}}{\Wrappedafterbreak}{\hbox{\char`\!}}}% 
            \lccode`\~`\/\lowercase{\def~}{\discretionary{\hbox{\char`\/}}{\Wrappedafterbreak}{\hbox{\char`\/}}}% 
            \catcode`\.\active
            \catcode`\,\active 
            \catcode`\;\active
            \catcode`\:\active
            \catcode`\?\active
            \catcode`\!\active
            \catcode`\/\active 
            \lccode`\~`\~ 	
        }
    \makeatother

    \let\OriginalVerbatim=\Verbatim
    \makeatletter
    \renewcommand{\Verbatim}[1][1]{%
        %\parskip\z@skip
        \sbox\Wrappedcontinuationbox {\Wrappedcontinuationsymbol}%
        \sbox\Wrappedvisiblespacebox {\FV@SetupFont\Wrappedvisiblespace}%
        \def\FancyVerbFormatLine ##1{\hsize\linewidth
            \vtop{\raggedright\hyphenpenalty\z@\exhyphenpenalty\z@
                \doublehyphendemerits\z@\finalhyphendemerits\z@
                \strut ##1\strut}%
        }%
        % If the linebreak is at a space, the latter will be displayed as visible
        % space at end of first line, and a continuation symbol starts next line.
        % Stretch/shrink are however usually zero for typewriter font.
        \def\FV@Space {%
            \nobreak\hskip\z@ plus\fontdimen3\font minus\fontdimen4\font
            \discretionary{\copy\Wrappedvisiblespacebox}{\Wrappedafterbreak}
            {\kern\fontdimen2\font}%
        }%
        
        % Allow breaks at special characters using \PYG... macros.
        \Wrappedbreaksatspecials
        % Breaks at punctuation characters . , ; ? ! and / need catcode=\active 	
        \OriginalVerbatim[#1,codes*=\Wrappedbreaksatpunct]%
    }
    \makeatother

    % Exact colors from NB
    \definecolor{incolor}{HTML}{303F9F}
    \definecolor{outcolor}{HTML}{D84315}
    \definecolor{cellborder}{HTML}{CFCFCF}
    \definecolor{cellbackground}{HTML}{F7F7F7}
    
    % prompt
    \makeatletter
    \newcommand{\boxspacing}{\kern\kvtcb@left@rule\kern\kvtcb@boxsep}
    \makeatother
    \newcommand{\prompt}[4]{
        \ttfamily\llap{{\color{#2}[#3]:\hspace{3pt}#4}}\vspace{-\baselineskip}
    }
    

    
    % Prevent overflowing lines due to hard-to-break entities
    \sloppy 
    % Setup hyperref package
    \hypersetup{
      breaklinks=true,  % so long urls are correctly broken across lines
      colorlinks=true,
      urlcolor=urlcolor,
      linkcolor=linkcolor,
      citecolor=citecolor,
      }
    % Slightly bigger margins than the latex defaults
    
    \geometry{verbose,tmargin=1in,bmargin=1in,lmargin=1in,rmargin=1in}
    
    

\begin{document}
    
    \maketitle
    
    

    
    \hypertarget{lab-2-advanced-circuits}{%
\section{Lab 2: Advanced Circuits}\label{lab-2-advanced-circuits}}

Welcome to Qiskit! Before starting with the exercises, please run the
cell below by pressing `shift' + `return'.

    \begin{tcolorbox}[breakable, size=fbox, boxrule=1pt, pad at break*=1mm,colback=cellbackground, colframe=cellborder]
\prompt{In}{incolor}{47}{\boxspacing}
\begin{Verbatim}[commandchars=\\\{\}]
\PY{k+kn}{import} \PY{n+nn}{numpy} \PY{k}{as} \PY{n+nn}{np}
\PY{k+kn}{from} \PY{n+nn}{qiskit}\PY{n+nn}{.}\PY{n+nn}{opflow} \PY{k+kn}{import} \PY{n}{I}\PY{p}{,} \PY{n}{X}\PY{p}{,} \PY{n}{Y}\PY{p}{,} \PY{n}{Z}\PY{p}{,} \PY{n}{MatrixEvolution}\PY{p}{,} \PY{n}{PauliTrotterEvolution}
\PY{k+kn}{from} \PY{n+nn}{qiskit}\PY{n+nn}{.}\PY{n+nn}{circuit} \PY{k+kn}{import} \PY{n}{Parameter}
\PY{k+kn}{from} \PY{n+nn}{qiskit} \PY{k+kn}{import} \PY{n}{QuantumCircuit}
\PY{k+kn}{from} \PY{n+nn}{qiskit} \PY{k+kn}{import} \PY{n}{Aer}
\PY{k+kn}{from} \PY{n+nn}{qiskit}\PY{n+nn}{.}\PY{n+nn}{compiler} \PY{k+kn}{import} \PY{n}{transpile}
\PY{k+kn}{import} \PY{n+nn}{qc\PYZus{}grader}
\end{Verbatim}
\end{tcolorbox}

    \hypertarget{ii.1-operators-and-qiskit-opflow}{%
\subsection{II.1: Operators and Qiskit
Opflow}\label{ii.1-operators-and-qiskit-opflow}}

In this notebook we will learn the fundamentals of Qiskit Opflow module,
this has advanced features that aim to bridge the gap from quantum
information theory to experiments. The Qiskit Opflow module will allow
us to quickly enter the realm of quantum simulation. Quoting the Qiskit
documentation:

``The Operator Flow is meant to serve as a lingua franca between the
theory and implementation of Quantum Algorithms \& Applications.
Meaning, the ultimate goal is that when theorists speak their theory in
the Operator Flow, they are speaking valid implementation, and when the
engineers speak their implementation in the Operator Flow, they are
speaking valid physical formalism.''

\href{https://qiskit.org/documentation/tutorials/operators/01_operator_flow.html}{Here}
you can find more documentation and examples on the module. To start,
let's get familiar with how to define operators.

\hypertarget{define-pauli-operators-for-one-qubit}{%
\subsubsection{1.) Define Pauli operators for one
qubit}\label{define-pauli-operators-for-one-qubit}}

Define operators for the four pauli matrices: \texttt{X}, \texttt{Y},
\texttt{Z}, \texttt{I}; and collect them in a list \texttt{pauli\_list}.

    \begin{tcolorbox}[breakable, size=fbox, boxrule=1pt, pad at break*=1mm,colback=cellbackground, colframe=cellborder]
\prompt{In}{incolor}{48}{\boxspacing}
\begin{Verbatim}[commandchars=\\\{\}]
\PY{c+c1}{\PYZsh{} Define which will contain the Paulis}
\PY{n}{pauli\PYZus{}list} \PY{o}{=} \PY{p}{[}\PY{n}{X}\PY{p}{,}\PY{n}{Y}\PY{p}{,}\PY{n}{Z}\PY{p}{,}\PY{n}{I}\PY{p}{]}

\PY{c+c1}{\PYZsh{} Define Paulis and add them to the list}
\PY{c+c1}{\PYZsh{}\PYZsh{}\PYZsh{}INSERT CODE BELOW THIS LINE}


\PY{c+c1}{\PYZsh{}\PYZsh{}\PYZsh{}DO NOT EDIT BELOW THIS LINE}

\PY{k}{for} \PY{n}{pauli} \PY{o+ow}{in} \PY{n}{pauli\PYZus{}list}\PY{p}{:}
    \PY{n+nb}{print}\PY{p}{(}\PY{n}{pauli}\PY{p}{,} \PY{l+s+s1}{\PYZsq{}}\PY{l+s+se}{\PYZbs{}n}\PY{l+s+s1}{\PYZsq{}}\PY{p}{)}
\end{Verbatim}
\end{tcolorbox}

    \begin{Verbatim}[commandchars=\\\{\}]
X

Y

Z

I

    \end{Verbatim}

    \begin{tcolorbox}[breakable, size=fbox, boxrule=1pt, pad at break*=1mm,colback=cellbackground, colframe=cellborder]
\prompt{In}{incolor}{49}{\boxspacing}
\begin{Verbatim}[commandchars=\\\{\}]
\PY{k+kn}{from} \PY{n+nn}{qc\PYZus{}grader}\PY{n+nn}{.}\PY{n+nn}{challenges}\PY{n+nn}{.}\PY{n+nn}{qgss\PYZus{}2022} \PY{k+kn}{import} \PY{n}{grade\PYZus{}lab2\PYZus{}ex1}

\PY{n}{grade\PYZus{}lab2\PYZus{}ex1}\PY{p}{(}\PY{n}{pauli\PYZus{}list}\PY{p}{)}
\end{Verbatim}
\end{tcolorbox}

    \begin{Verbatim}[commandchars=\\\{\}]
Submitting your answer. Please wait{\ldots}
Congratulations 🎉! Your answer is correct and has been submitted.
    \end{Verbatim}

    There's a few operations that we can do on operators which are
implemented in Qiskit Opflow. For example, we can rescale an operator by
a scalar factor using \texttt{*}, we can compose operators using
\texttt{@} and we can take the tensor product of operators using
\texttt{\^{}}. In the following, let us try to use these operations.
Note that we need to be careful with the operations' precedences as
python evaluates \texttt{+} before \texttt{\^{}} and that may change the
intended value of an expression. For example, \texttt{I\^{}X+X\^{}I} is
actually interpreted as \texttt{I\^{}(X+X)\^{}I=2(I\^{}X\^{}I)}.
Therefore the use of parenthesis is strongly recommended to avoid these
types of errors. Also, keep in mind that the imaginary unit i is defined
as \texttt{1j} in Python.

\hypertarget{define-the-ladder-operator-hatsigma-frac-hatsigma_x-i-hatsigma_y2-and-hatsigma--frachatsigma_x---i-hatsigma_y2}{%
\subsubsection{\texorpdfstring{2.) Define the ladder operator:
\(\hat{\sigma}^{+} = \frac{ \hat{\sigma}_x + i \hat{\sigma}_y}{2}\) and
\(\hat{\sigma}^{-} = \frac{\hat{\sigma}_x - i \hat{\sigma}_y}{2}\)}{2.) Define the ladder operator: \textbackslash hat\{\textbackslash sigma\}\^{}\{+\} = \textbackslash frac\{ \textbackslash hat\{\textbackslash sigma\}\_x + i \textbackslash hat\{\textbackslash sigma\}\_y\}\{2\} and \textbackslash hat\{\textbackslash sigma\}\^{}\{-\} = \textbackslash frac\{\textbackslash hat\{\textbackslash sigma\}\_x - i \textbackslash hat\{\textbackslash sigma\}\_y\}\{2\}}}\label{define-the-ladder-operator-hatsigma-frac-hatsigma_x-i-hatsigma_y2-and-hatsigma--frachatsigma_x---i-hatsigma_y2}}

    \begin{tcolorbox}[breakable, size=fbox, boxrule=1pt, pad at break*=1mm,colback=cellbackground, colframe=cellborder]
\prompt{In}{incolor}{50}{\boxspacing}
\begin{Verbatim}[commandchars=\\\{\}]
\PY{c+c1}{\PYZsh{} Define list of ladder operators}
\PY{n}{ladder\PYZus{}operator\PYZus{}list} \PY{o}{=} \PY{p}{[}\PY{l+m+mf}{0.5} \PY{o}{*} \PY{p}{(}\PY{n}{X}\PY{o}{+}\PY{l+m+mi}{1}\PY{n}{j} \PY{o}{*} \PY{n}{Y}\PY{p}{)}\PY{p}{,} \PY{l+m+mf}{0.5} \PY{o}{*} \PY{p}{(}\PY{n}{X}\PY{o}{\PYZhy{}}\PY{l+m+mi}{1}\PY{n}{j} \PY{o}{*} \PY{n}{Y}\PY{p}{)}\PY{p}{]}

\PY{c+c1}{\PYZsh{} Define ladder operators and add the to the list}
\PY{c+c1}{\PYZsh{}\PYZsh{}\PYZsh{}INSERT CODE BELOW THIS LINE}


\PY{c+c1}{\PYZsh{}\PYZsh{}\PYZsh{}DO NOT EDIT BELOW THIS LINE}

\PY{k}{for} \PY{n}{ladder\PYZus{}operator} \PY{o+ow}{in} \PY{n}{ladder\PYZus{}operator\PYZus{}list}\PY{p}{:}
    \PY{n+nb}{print}\PY{p}{(}\PY{n}{ladder\PYZus{}operator}\PY{p}{,} \PY{l+s+s1}{\PYZsq{}}\PY{l+s+se}{\PYZbs{}n}\PY{l+s+s1}{\PYZsq{}}\PY{p}{)}
\end{Verbatim}
\end{tcolorbox}

    \begin{Verbatim}[commandchars=\\\{\}]
0.5 * X
+ 0.5j * Y

0.5 * X
+ -0.5j * Y

    \end{Verbatim}

    \begin{tcolorbox}[breakable, size=fbox, boxrule=1pt, pad at break*=1mm,colback=cellbackground, colframe=cellborder]
\prompt{In}{incolor}{51}{\boxspacing}
\begin{Verbatim}[commandchars=\\\{\}]
\PY{k+kn}{from} \PY{n+nn}{qc\PYZus{}grader}\PY{n+nn}{.}\PY{n+nn}{challenges}\PY{n+nn}{.}\PY{n+nn}{qgss\PYZus{}2022} \PY{k+kn}{import} \PY{n}{grade\PYZus{}lab2\PYZus{}ex2}

\PY{n}{grade\PYZus{}lab2\PYZus{}ex2}\PY{p}{(}\PY{n}{ladder\PYZus{}operator\PYZus{}list}\PY{p}{)}
\end{Verbatim}
\end{tcolorbox}

    \begin{Verbatim}[commandchars=\\\{\}]
Submitting your answer. Please wait{\ldots}
Congratulations 🎉! Your answer is correct and has been submitted.
    \end{Verbatim}

    We can take the operators defined in Qiskit Opflow and translate them
into other representation. For example the \texttt{to\_matrix()} method
of an Operator object allows us to retrieve the matrix representation of
the operator (as a numpy array)

\hypertarget{obtain-the-matrix-representation-of-the-pauli-operators-sigma_x-sigma_y-sigma_z-and-identity}{%
\subsubsection{\texorpdfstring{3.) Obtain the matrix representation of
the pauli operators (\texttt{sigma\_X}, \texttt{sigma\_Y},
\texttt{sigma\_Z} and \texttt{identity}
)}{3.) Obtain the matrix representation of the pauli operators (sigma\_X, sigma\_Y, sigma\_Z and identity )}}\label{obtain-the-matrix-representation-of-the-pauli-operators-sigma_x-sigma_y-sigma_z-and-identity}}

Please submit the result as a list with the operators ordered as above.

    \begin{tcolorbox}[breakable, size=fbox, boxrule=1pt, pad at break*=1mm,colback=cellbackground, colframe=cellborder]
\prompt{In}{incolor}{52}{\boxspacing}
\begin{Verbatim}[commandchars=\\\{\}]
\PY{c+c1}{\PYZsh{} Define list which will contain the matrices representing the Pauli operators}
\PY{n}{matrix\PYZus{}sigma\PYZus{}list} \PY{o}{=} \PY{p}{[}\PY{n}{X}\PY{o}{.}\PY{n}{to\PYZus{}matrix}\PY{p}{(}\PY{p}{)}\PY{p}{,} \PY{n}{Y}\PY{o}{.}\PY{n}{to\PYZus{}matrix}\PY{p}{(}\PY{p}{)}\PY{p}{,} \PY{n}{Z}\PY{o}{.}\PY{n}{to\PYZus{}matrix}\PY{p}{(}\PY{p}{)}\PY{p}{,} \PY{n}{I}\PY{o}{.}\PY{n}{to\PYZus{}matrix}\PY{p}{(}\PY{p}{)}\PY{p}{]}

\PY{c+c1}{\PYZsh{} Add matrix representation of Paulis to the list}
\PY{c+c1}{\PYZsh{}\PYZsh{}\PYZsh{}INSERT CODE BELOW THIS LINE}


\PY{c+c1}{\PYZsh{}\PYZsh{}\PYZsh{}DO NOT EDIT BELOW THIS LINE}

\PY{k}{for} \PY{n}{matrix\PYZus{}sigma} \PY{o+ow}{in} \PY{n}{matrix\PYZus{}sigma\PYZus{}list}\PY{p}{:}
    \PY{n+nb}{print}\PY{p}{(}\PY{n}{matrix\PYZus{}sigma}\PY{p}{,} \PY{l+s+s1}{\PYZsq{}}\PY{l+s+se}{\PYZbs{}n}\PY{l+s+s1}{\PYZsq{}}\PY{p}{)}
\end{Verbatim}
\end{tcolorbox}

    \begin{Verbatim}[commandchars=\\\{\}]
[[0.+0.j 1.+0.j]
 [1.+0.j 0.+0.j]]

[[0.+0.j 0.-1.j]
 [0.+1.j 0.+0.j]]

[[ 1.+0.j  0.+0.j]
 [ 0.+0.j -1.+0.j]]

[[1.+0.j 0.+0.j]
 [0.+0.j 1.+0.j]]

    \end{Verbatim}

    \begin{tcolorbox}[breakable, size=fbox, boxrule=1pt, pad at break*=1mm,colback=cellbackground, colframe=cellborder]
\prompt{In}{incolor}{53}{\boxspacing}
\begin{Verbatim}[commandchars=\\\{\}]
\PY{k+kn}{from} \PY{n+nn}{qc\PYZus{}grader}\PY{n+nn}{.}\PY{n+nn}{challenges}\PY{n+nn}{.}\PY{n+nn}{qgss\PYZus{}2022} \PY{k+kn}{import} \PY{n}{grade\PYZus{}lab2\PYZus{}ex3}

\PY{n}{grade\PYZus{}lab2\PYZus{}ex3}\PY{p}{(}\PY{n}{matrix\PYZus{}sigma\PYZus{}list}\PY{p}{)}
\end{Verbatim}
\end{tcolorbox}

    \begin{Verbatim}[commandchars=\\\{\}]
Submitting your answer. Please wait{\ldots}
Congratulations 🎉! Your answer is correct and has been submitted.
    \end{Verbatim}

    We can also generate a circuit representation of the operator using the
\texttt{to\_circuit()} method

\hypertarget{obtain-the-circuit-representation-of-the-pauli-operators-sigma_x-sigma_y-sigma_z-and-identity}{%
\subsubsection{\texorpdfstring{4.) Obtain the circuit representation of
the pauli operators (\texttt{sigma\_X}, \texttt{sigma\_Y},
\texttt{sigma\_Z} and \texttt{identity}
)}{4.) Obtain the circuit representation of the pauli operators (sigma\_X, sigma\_Y, sigma\_Z and identity )}}\label{obtain-the-circuit-representation-of-the-pauli-operators-sigma_x-sigma_y-sigma_z-and-identity}}

Please submit the result as a list with the operators ordered as above.

    \begin{tcolorbox}[breakable, size=fbox, boxrule=1pt, pad at break*=1mm,colback=cellbackground, colframe=cellborder]
\prompt{In}{incolor}{54}{\boxspacing}
\begin{Verbatim}[commandchars=\\\{\}]
\PY{c+c1}{\PYZsh{} Define a list which will contain the circuit representation of the Paulis}
\PY{n}{circuit\PYZus{}sigma\PYZus{}list} \PY{o}{=} \PY{p}{[}\PY{n}{X}\PY{o}{.}\PY{n}{to\PYZus{}circuit}\PY{p}{(}\PY{p}{)}\PY{p}{,} \PY{n}{Y}\PY{o}{.}\PY{n}{to\PYZus{}circuit}\PY{p}{(}\PY{p}{)}\PY{p}{,} \PY{n}{Z}\PY{o}{.}\PY{n}{to\PYZus{}circuit}\PY{p}{(}\PY{p}{)}\PY{p}{,} \PY{n}{I}\PY{o}{.}\PY{n}{to\PYZus{}circuit}\PY{p}{(}\PY{p}{)}\PY{p}{]}

\PY{c+c1}{\PYZsh{} Add circuits to list}
\PY{c+c1}{\PYZsh{}\PYZsh{}\PYZsh{}INSERT CODE BELOW THIS LINE}


\PY{c+c1}{\PYZsh{}\PYZsh{}\PYZsh{}DO NOT EDIT BELOW THIS LINE}

\PY{k}{for} \PY{n}{circuit} \PY{o+ow}{in} \PY{n}{circuit\PYZus{}sigma\PYZus{}list}\PY{p}{:}
    \PY{n+nb}{print}\PY{p}{(}\PY{n}{circuit}\PY{p}{,} \PY{l+s+s1}{\PYZsq{}}\PY{l+s+se}{\PYZbs{}n}\PY{l+s+s1}{\PYZsq{}}\PY{p}{)}
\end{Verbatim}
\end{tcolorbox}

    \begin{Verbatim}[commandchars=\\\{\}]
   ┌───┐
q: ┤ X ├
   └───┘

   ┌───┐
q: ┤ Y ├
   └───┘

   ┌───┐
q: ┤ Z ├
   └───┘


q:


    \end{Verbatim}

    \begin{tcolorbox}[breakable, size=fbox, boxrule=1pt, pad at break*=1mm,colback=cellbackground, colframe=cellborder]
\prompt{In}{incolor}{55}{\boxspacing}
\begin{Verbatim}[commandchars=\\\{\}]
\PY{k+kn}{from} \PY{n+nn}{qc\PYZus{}grader}\PY{n+nn}{.}\PY{n+nn}{challenges}\PY{n+nn}{.}\PY{n+nn}{qgss\PYZus{}2022} \PY{k+kn}{import} \PY{n}{grade\PYZus{}lab2\PYZus{}ex4}

\PY{n}{grade\PYZus{}lab2\PYZus{}ex4}\PY{p}{(}\PY{n}{circuit\PYZus{}sigma\PYZus{}list}\PY{p}{)}
\end{Verbatim}
\end{tcolorbox}

    \begin{Verbatim}[commandchars=\\\{\}]
Submitting your answer. Please wait{\ldots}
Congratulations 🎉! Your answer is correct and has been submitted.
    \end{Verbatim}

    \hypertarget{ii.2-simulating-time-evolution-with-quantum-circuits}{%
\subsection{II.2: Simulating Time-Evolution with Quantum
Circuits}\label{ii.2-simulating-time-evolution-with-quantum-circuits}}

Now that we are a little more familiar with the syntax of the Qiskit
Opflow module we can put this knowledge to use to build the first
quantum circuit simulating the dynamics (or time-evolution) of a system
described by a given Hamiltonian. As a first step, let us introduce
parametrized circuits. Below we'll create a circuit with a parametrized
rotation with an angle \(\theta\). The goal is not to directly use
parametrized rotations but to understand how Qiskit's quantum circuit
can accept parameters whose values will be defined later on. We'll need
that to create circuits that represent time-evolution operators with a
parametrized value for the time.

\hypertarget{create-a-circuit-with-a-parametrized-rx-rotation-of-an-angle-theta}{%
\subsubsection{\texorpdfstring{1.) Create a circuit with a parametrized
RX rotation of an angle
\(\theta\)}{1.) Create a circuit with a parametrized RX rotation of an angle \textbackslash theta}}\label{create-a-circuit-with-a-parametrized-rx-rotation-of-an-angle-theta}}

    \begin{tcolorbox}[breakable, size=fbox, boxrule=1pt, pad at break*=1mm,colback=cellbackground, colframe=cellborder]
\prompt{In}{incolor}{56}{\boxspacing}
\begin{Verbatim}[commandchars=\\\{\}]
\PY{c+c1}{\PYZsh{} Define a variable theta to be a parameter with name \PYZsq{}theta\PYZsq{}}
\PY{n}{theta} \PY{o}{=} \PY{n}{Parameter}\PY{p}{(}\PY{l+s+s1}{\PYZsq{}}\PY{l+s+s1}{theta}\PY{l+s+s1}{\PYZsq{}}\PY{p}{)}
\PY{c+c1}{\PYZsh{} Set number of qubits to 1}
\PY{n}{qubits\PYZus{}count} \PY{o}{=} \PY{l+m+mi}{1}
\PY{c+c1}{\PYZsh{} Initialize a quantum circuit with one qubit}
\PY{n}{quantum\PYZus{}circuit} \PY{o}{=} \PY{n}{QuantumCircuit}\PY{p}{(}\PY{n}{qubits\PYZus{}count}\PY{p}{)}

\PY{c+c1}{\PYZsh{} Add a parametrized RX rotation on the qubit}
\PY{c+c1}{\PYZsh{}\PYZsh{}\PYZsh{}INSERT CODE BELOW THIS LINE}
\PY{n}{quantum\PYZus{}circuit}\PY{o}{.}\PY{n}{rx}\PY{p}{(}\PY{n}{theta}\PY{p}{,} \PY{l+m+mi}{0}\PY{p}{)}

\PY{c+c1}{\PYZsh{}\PYZsh{}\PYZsh{}DO NOT EDIT BELOW THIS LINE}

\PY{n+nb}{print}\PY{p}{(}\PY{n}{quantum\PYZus{}circuit}\PY{p}{)}
\end{Verbatim}
\end{tcolorbox}

    \begin{Verbatim}[commandchars=\\\{\}]
   ┌───────────┐
q: ┤ Rx(theta) ├
   └───────────┘
    \end{Verbatim}

    \begin{tcolorbox}[breakable, size=fbox, boxrule=1pt, pad at break*=1mm,colback=cellbackground, colframe=cellborder]
\prompt{In}{incolor}{57}{\boxspacing}
\begin{Verbatim}[commandchars=\\\{\}]
\PY{k+kn}{from} \PY{n+nn}{qc\PYZus{}grader}\PY{n+nn}{.}\PY{n+nn}{challenges}\PY{n+nn}{.}\PY{n+nn}{qgss\PYZus{}2022} \PY{k+kn}{import} \PY{n}{grade\PYZus{}lab2\PYZus{}ex5}

\PY{n}{grade\PYZus{}lab2\PYZus{}ex5}\PY{p}{(}\PY{n}{quantum\PYZus{}circuit}\PY{p}{)}
\end{Verbatim}
\end{tcolorbox}

    \begin{Verbatim}[commandchars=\\\{\}]
Submitting your answer. Please wait{\ldots}
Congratulations 🎉! Your answer is correct and has been submitted.
    \end{Verbatim}

    This creates a circuit where the parameter of the RX gate is a
placeholder which is waiting for a value. Once we decide on the value of
\(\theta\), we can bind it to the circuit using the
\texttt{bind\_parameters(\{parameter:\ parameter\_value\})} method of
the \texttt{QuantumCircuit} object.

\hypertarget{bind-numerical-value-of-the-angle-theta}{%
\subsubsection{\texorpdfstring{2.) Bind numerical value of the angle
\(\theta\)}{2.) Bind numerical value of the angle \textbackslash theta}}\label{bind-numerical-value-of-the-angle-theta}}

    \begin{tcolorbox}[breakable, size=fbox, boxrule=1pt, pad at break*=1mm,colback=cellbackground, colframe=cellborder]
\prompt{In}{incolor}{58}{\boxspacing}
\begin{Verbatim}[commandchars=\\\{\}]
\PY{c+c1}{\PYZsh{} Set the value of the parameter}
\PY{n}{theta\PYZus{}value} \PY{o}{=} \PY{n}{np}\PY{o}{.}\PY{n}{pi}

\PY{c+c1}{\PYZsh{} Bind the value to the parametrized circuit}
\PY{c+c1}{\PYZsh{}\PYZsh{}\PYZsh{}INSERT CODE BELOW THIS LINE}
\PY{n}{quantum\PYZus{}circuit}\PY{o}{.}\PY{n}{assign\PYZus{}parameters}\PY{p}{(}\PY{p}{\PYZob{}}\PY{n}{theta}\PY{p}{:}\PY{n}{theta\PYZus{}value}\PY{p}{\PYZcb{}}\PY{p}{,} \PY{n}{inplace}\PY{o}{=}\PY{k+kc}{True}\PY{p}{)}

\PY{c+c1}{\PYZsh{}\PYZsh{}\PYZsh{}DO NOT EDIT BELOW THIS LINE}

\PY{n+nb}{print}\PY{p}{(}\PY{n}{quantum\PYZus{}circuit}\PY{p}{)}
\end{Verbatim}
\end{tcolorbox}

    \begin{Verbatim}[commandchars=\\\{\}]
   ┌───────┐
q: ┤ Rx(π) ├
   └───────┘
    \end{Verbatim}

    \begin{tcolorbox}[breakable, size=fbox, boxrule=1pt, pad at break*=1mm,colback=cellbackground, colframe=cellborder]
\prompt{In}{incolor}{59}{\boxspacing}
\begin{Verbatim}[commandchars=\\\{\}]
\PY{k+kn}{from} \PY{n+nn}{qc\PYZus{}grader}\PY{n+nn}{.}\PY{n+nn}{challenges}\PY{n+nn}{.}\PY{n+nn}{qgss\PYZus{}2022} \PY{k+kn}{import} \PY{n}{grade\PYZus{}lab2\PYZus{}ex6}

\PY{n}{grade\PYZus{}lab2\PYZus{}ex6}\PY{p}{(}\PY{n}{quantum\PYZus{}circuit}\PY{p}{)}
\end{Verbatim}
\end{tcolorbox}

    \begin{Verbatim}[commandchars=\\\{\}]
Submitting your answer. Please wait{\ldots}
Congratulations 🎉! Your answer is correct and has been submitted.
    \end{Verbatim}

    Let us start to prepare the building blocks we'll need to calculate the
time-evolution of a quantum system using a quantum computer. First let's
define the Hamiltonian of the system to be the Heisenberg Hamiltonian
for two qubits:

\[ \hat{H} = \frac{1}{2} \left( \hat{I}\otimes \hat{I} +  \hat{\sigma}_x \otimes \hat{\sigma}_x + \hat{\sigma}_y \otimes \hat{\sigma}_y + \hat{\sigma}_z \otimes \hat{\sigma}_z \right) \]

\hypertarget{define-the-heisenberg-hamiltonian-using-qiskit-opflow}{%
\subsubsection{3.) Define the Heisenberg Hamiltonian using Qiskit
Opflow}\label{define-the-heisenberg-hamiltonian-using-qiskit-opflow}}

    \begin{tcolorbox}[breakable, size=fbox, boxrule=1pt, pad at break*=1mm,colback=cellbackground, colframe=cellborder]
\prompt{In}{incolor}{60}{\boxspacing}
\begin{Verbatim}[commandchars=\\\{\}]
\PY{c+c1}{\PYZsh{} Use the formula above to define the Hamiltonian operator}
\PY{c+c1}{\PYZsh{}\PYZsh{}\PYZsh{}INSERT CODE BELOW THIS LINE}

\PY{n}{H} \PY{o}{=} \PY{l+m+mf}{0.5}\PY{o}{*}\PY{p}{(}\PY{p}{(}\PY{n}{I}\PY{o}{\PYZca{}}\PY{n}{I}\PY{p}{)}\PY{o}{+}\PY{p}{(}\PY{n}{X}\PY{o}{\PYZca{}}\PY{n}{X}\PY{p}{)}\PY{o}{+}\PY{p}{(}\PY{n}{Y}\PY{o}{\PYZca{}}\PY{n}{Y}\PY{p}{)}\PY{o}{+}\PY{p}{(}\PY{n}{Z}\PY{o}{\PYZca{}}\PY{n}{Z}\PY{p}{)}\PY{p}{)}
\PY{c+c1}{\PYZsh{}\PYZsh{}\PYZsh{}DO NOT EDIT BELOW THIS LINE}

\PY{c+c1}{\PYZsh{} Get its matrix representation}
\PY{n}{H\PYZus{}matrix} \PY{o}{=} \PY{n}{H}\PY{o}{.}\PY{n}{to\PYZus{}matrix}\PY{p}{(}\PY{p}{)}

\PY{n+nb}{print}\PY{p}{(}\PY{n}{H\PYZus{}matrix}\PY{p}{)}
\end{Verbatim}
\end{tcolorbox}

    \begin{Verbatim}[commandchars=\\\{\}]
[[1.+0.j 0.+0.j 0.+0.j 0.+0.j]
 [0.+0.j 0.+0.j 1.+0.j 0.+0.j]
 [0.+0.j 1.+0.j 0.+0.j 0.+0.j]
 [0.+0.j 0.+0.j 0.+0.j 1.+0.j]]
    \end{Verbatim}

    \begin{tcolorbox}[breakable, size=fbox, boxrule=1pt, pad at break*=1mm,colback=cellbackground, colframe=cellborder]
\prompt{In}{incolor}{61}{\boxspacing}
\begin{Verbatim}[commandchars=\\\{\}]
\PY{k+kn}{from} \PY{n+nn}{qc\PYZus{}grader}\PY{n+nn}{.}\PY{n+nn}{challenges}\PY{n+nn}{.}\PY{n+nn}{qgss\PYZus{}2022} \PY{k+kn}{import} \PY{n}{grade\PYZus{}lab2\PYZus{}ex7}

\PY{n}{grade\PYZus{}lab2\PYZus{}ex7}\PY{p}{(}\PY{n}{H\PYZus{}matrix}\PY{p}{)}
\end{Verbatim}
\end{tcolorbox}

    \begin{Verbatim}[commandchars=\\\{\}]
Submitting your answer. Please wait{\ldots}
Congratulations 🎉! Your answer is correct and has been submitted.
    \end{Verbatim}

    Next, let's create a quantum circuit for time evolution! We'll
parametrize the time \texttt{t} with a Qiskit \texttt{Parameter} and
exponentiate the Heisenberg Hamiltonian with the Qiskit Opflow method
\texttt{exp\_i()} which implements the corressponding time-evolution
operator \(e^{-i \hat{H} t}\)

\hypertarget{define-the-time-evolution-operator-for-the-heisenberg-hamiltonian-hath-and-the-time-step-t}{%
\subsubsection{\texorpdfstring{4.) Define the time evolution operator
for the Heisenberg Hamiltonian \(\hat{H}\) and the time step
\(t\)}{4.) Define the time evolution operator for the Heisenberg Hamiltonian \textbackslash hat\{H\} and the time step t}}\label{define-the-time-evolution-operator-for-the-heisenberg-hamiltonian-hath-and-the-time-step-t}}

    \begin{tcolorbox}[breakable, size=fbox, boxrule=1pt, pad at break*=1mm,colback=cellbackground, colframe=cellborder]
\prompt{In}{incolor}{62}{\boxspacing}
\begin{Verbatim}[commandchars=\\\{\}]
\PY{c+c1}{\PYZsh{} Define a parameter t for the time in the time evolution operator}
\PY{n}{t} \PY{o}{=} \PY{n}{Parameter}\PY{p}{(}\PY{l+s+s1}{\PYZsq{}}\PY{l+s+s1}{t}\PY{l+s+s1}{\PYZsq{}}\PY{p}{)}

\PY{c+c1}{\PYZsh{} Follow the instructions above to define a time\PYZhy{}evolution operator}
\PY{c+c1}{\PYZsh{}\PYZsh{}\PYZsh{}INSERT CODE BELOW THIS LINE}
\PY{n}{time\PYZus{}evolution\PYZus{}operator} \PY{o}{=} \PY{p}{(}\PY{n}{H}\PY{o}{*}\PY{n}{t}\PY{p}{)}\PY{o}{.}\PY{n}{exp\PYZus{}i}\PY{p}{(}\PY{p}{)}

\PY{c+c1}{\PYZsh{}\PYZsh{}\PYZsh{}DO NOT EDIT BELOW THIS LINE}

\PY{n+nb}{print}\PY{p}{(}\PY{n}{time\PYZus{}evolution\PYZus{}operator}\PY{p}{)}
\end{Verbatim}
\end{tcolorbox}

    \begin{Verbatim}[commandchars=\\\{\}]
e\^{}(-i*1.0*t * (
  0.5 * II
  + 0.5 * XX
  + 0.5 * YY
  + 0.5 * ZZ
))
    \end{Verbatim}

    \begin{tcolorbox}[breakable, size=fbox, boxrule=1pt, pad at break*=1mm,colback=cellbackground, colframe=cellborder]
\prompt{In}{incolor}{63}{\boxspacing}
\begin{Verbatim}[commandchars=\\\{\}]
\PY{k+kn}{from} \PY{n+nn}{qc\PYZus{}grader}\PY{n+nn}{.}\PY{n+nn}{challenges}\PY{n+nn}{.}\PY{n+nn}{qgss\PYZus{}2022} \PY{k+kn}{import} \PY{n}{grade\PYZus{}lab2\PYZus{}ex8}

\PY{n}{grade\PYZus{}lab2\PYZus{}ex8}\PY{p}{(}\PY{n}{time\PYZus{}evolution\PYZus{}operator}\PY{p}{)}
\end{Verbatim}
\end{tcolorbox}

    \begin{Verbatim}[commandchars=\\\{\}]
Submitting your answer. Please wait{\ldots}
Congratulations 🎉! Your answer is correct and has been submitted.
    \end{Verbatim}

    We can then generate a circuit which implements the necessary operations
that compute the time-evolution operator for a given evolution time.
First, let's try to do this exactly with the \texttt{MatrxEvolution}
class of Qiskit Opflow.

\hypertarget{use-matrixevolution-to-calculate-the-exact-exponentiation-at-time-t}{%
\subsubsection{\texorpdfstring{5.) Use \texttt{MatrixEvolution} to
calculate the exact exponentiation at time
\(t\)}{5.) Use MatrixEvolution to calculate the exact exponentiation at time t}}\label{use-matrixevolution-to-calculate-the-exact-exponentiation-at-time-t}}

*Hint: First you'll need to instantiate a \texttt{MatrixEvolution()}
object. This object has a method called \texttt{convert(operator)} which
takes a time-evolution operator and generates a quantum circuit
implementing the operation. Finally, you'll need to bind the value of
the evolution time to the circuit.

    \begin{tcolorbox}[breakable, size=fbox, boxrule=1pt, pad at break*=1mm,colback=cellbackground, colframe=cellborder]
\prompt{In}{incolor}{64}{\boxspacing}
\begin{Verbatim}[commandchars=\\\{\}]
\PY{c+c1}{\PYZsh{} Set a total time for the time evolution}
\PY{n}{evolution\PYZus{}time} \PY{o}{=} \PY{l+m+mf}{0.5}

\PY{c+c1}{\PYZsh{} Instantiate a MatrixEvolution() object to convert the time evolution operator}
\PY{c+c1}{\PYZsh{}  and bind the value for the time parameter}
\PY{c+c1}{\PYZsh{}\PYZsh{}\PYZsh{}INSERT CODE BELOW THIS LINE}

\PY{n}{me} \PY{o}{=} \PY{n}{MatrixEvolution}\PY{p}{(}\PY{p}{)}
\PY{n}{bound\PYZus{}matrix\PYZus{}exponentiation\PYZus{}circuit} \PY{o}{=} \PY{n}{me}\PY{o}{.}\PY{n}{convert}\PY{p}{(}\PY{n}{time\PYZus{}evolution\PYZus{}operator}\PY{p}{)}\PY{o}{.}\PY{n}{assign\PYZus{}parameters}\PY{p}{(}\PY{p}{\PYZob{}}\PY{n}{t}\PY{p}{:}\PY{n}{evolution\PYZus{}time}\PY{p}{\PYZcb{}}\PY{p}{)}

\PY{c+c1}{\PYZsh{}\PYZsh{}\PYZsh{}DO NOT EDIT BELOW THIS LINE}

\PY{n+nb}{print}\PY{p}{(}\PY{n}{bound\PYZus{}matrix\PYZus{}exponentiation\PYZus{}circuit}\PY{p}{)}
\end{Verbatim}
\end{tcolorbox}

    \begin{Verbatim}[commandchars=\\\{\}]
Evolved Hamiltonian is not composed of only MatrixOps, converting to Matrix
representation, which can be expensive.
    \end{Verbatim}

    \begin{Verbatim}[commandchars=\\\{\}]
     ┌──────────────┐
q\_0: ┤0             ├
     │  Hamiltonian │
q\_1: ┤1             ├
     └──────────────┘
    \end{Verbatim}

    \begin{tcolorbox}[breakable, size=fbox, boxrule=1pt, pad at break*=1mm,colback=cellbackground, colframe=cellborder]
\prompt{In}{incolor}{65}{\boxspacing}
\begin{Verbatim}[commandchars=\\\{\}]
\PY{k+kn}{from} \PY{n+nn}{qc\PYZus{}grader}\PY{n+nn}{.}\PY{n+nn}{challenges}\PY{n+nn}{.}\PY{n+nn}{qgss\PYZus{}2022} \PY{k+kn}{import} \PY{n}{grade\PYZus{}lab2\PYZus{}ex9}

\PY{n}{grade\PYZus{}lab2\PYZus{}ex9}\PY{p}{(}\PY{n}{bound\PYZus{}matrix\PYZus{}exponentiation\PYZus{}circuit}\PY{p}{)}
\end{Verbatim}
\end{tcolorbox}

    \begin{Verbatim}[commandchars=\\\{\}]
Submitting your answer. Please wait{\ldots}
Congratulations 🎉! Your answer is correct and has been submitted.
    \end{Verbatim}

    As a last step, let us also generate the circuit corresponding to the
time-evolution operator calculated using the Trotter-Suzuki
decomposition. For this we'll use the \texttt{PauliTrotterEvolution}
class in the same way we've used the \texttt{MatrixEvolution()} one.

\hypertarget{use-paulitrotterevolution-to-calculate-the-approximate-exponentiation-for-a-small-time-step}{%
\subsubsection{\texorpdfstring{6.) Use \texttt{PauliTrotterEvolution} to
calculate the approximate exponentiation for a small time
step}{6.) Use PauliTrotterEvolution to calculate the approximate exponentiation for a small time step}}\label{use-paulitrotterevolution-to-calculate-the-approximate-exponentiation-for-a-small-time-step}}

    \begin{tcolorbox}[breakable, size=fbox, boxrule=1pt, pad at break*=1mm,colback=cellbackground, colframe=cellborder]
\prompt{In}{incolor}{66}{\boxspacing}
\begin{Verbatim}[commandchars=\\\{\}]
\PY{c+c1}{\PYZsh{} Define a value for the duration of the time\PYZhy{}step}
\PY{n}{time\PYZus{}step\PYZus{}value} \PY{o}{=} \PY{l+m+mf}{0.1}

\PY{c+c1}{\PYZsh{} Instantiate a PauliTrotterEvolution() object and convert the time\PYZhy{}evolution operator}
\PY{c+c1}{\PYZsh{} to then bind the value of the time step}
\PY{c+c1}{\PYZsh{}\PYZsh{}\PYZsh{}INSERT CODE BELOW THIS LINE}

\PY{n}{bound\PYZus{}trotter\PYZus{}exponentiation\PYZus{}circuit} \PY{o}{=} \PY{n}{PauliTrotterEvolution}\PY{p}{(}\PY{p}{)}\PY{o}{.}\PY{n}{convert}\PY{p}{(}\PY{n}{time\PYZus{}evolution\PYZus{}operator}\PY{p}{)}\PY{o}{.}\PY{n}{assign\PYZus{}parameters}\PY{p}{(}\PY{p}{\PYZob{}}\PY{n}{t}\PY{p}{:}\PY{n}{time\PYZus{}step\PYZus{}value}\PY{p}{\PYZcb{}}\PY{p}{)}

\PY{c+c1}{\PYZsh{}\PYZsh{}\PYZsh{}DO NOT EDIT BELOW THIS LINE}

\PY{n+nb}{print}\PY{p}{(}\PY{n}{bound\PYZus{}trotter\PYZus{}exponentiation\PYZus{}circuit}\PY{p}{)}
\end{Verbatim}
\end{tcolorbox}

    \begin{Verbatim}[commandchars=\\\{\}]
     ┌────────────────────────────────────┐
q\_0: ┤0                                   ├
     │  exp(-it (II + XX + YY + ZZ))(0.1) │
q\_1: ┤1                                   ├
     └────────────────────────────────────┘
    \end{Verbatim}

    \begin{tcolorbox}[breakable, size=fbox, boxrule=1pt, pad at break*=1mm,colback=cellbackground, colframe=cellborder]
\prompt{In}{incolor}{67}{\boxspacing}
\begin{Verbatim}[commandchars=\\\{\}]
\PY{k+kn}{from} \PY{n+nn}{qc\PYZus{}grader}\PY{n+nn}{.}\PY{n+nn}{challenges}\PY{n+nn}{.}\PY{n+nn}{qgss\PYZus{}2022} \PY{k+kn}{import} \PY{n}{grade\PYZus{}lab2\PYZus{}ex10}

\PY{n}{grade\PYZus{}lab2\PYZus{}ex10}\PY{p}{(}\PY{n}{bound\PYZus{}trotter\PYZus{}exponentiation\PYZus{}circuit}\PY{p}{)}
\end{Verbatim}
\end{tcolorbox}

    \begin{Verbatim}[commandchars=\\\{\}]
Submitting your answer. Please wait{\ldots}
Congratulations 🎉! Your answer is correct and has been submitted.
    \end{Verbatim}

    The full evolution can then be obtained by putting together several
Trotter steps up to the desired evolution time. For each Trotter step
you can compose the single-step Trotter using the \texttt{@} operator.

\hypertarget{concatenate-several-trotter-steps-to-generate-the-desired-evolution}{%
\subsubsection{7.) Concatenate several Trotter steps to generate the
desired
evolution}\label{concatenate-several-trotter-steps-to-generate-the-desired-evolution}}

    \begin{tcolorbox}[breakable, size=fbox, boxrule=1pt, pad at break*=1mm,colback=cellbackground, colframe=cellborder]
\prompt{In}{incolor}{68}{\boxspacing}
\begin{Verbatim}[commandchars=\\\{\}]
\PY{c+c1}{\PYZsh{} Define the number of steps needed to reach the previously set total time\PYZhy{}evolution}
\PY{n}{steps} \PY{o}{=} \PY{n+nb}{int}\PY{p}{(}\PY{n}{evolution\PYZus{}time}\PY{o}{/}\PY{n}{time\PYZus{}step\PYZus{}value}\PY{p}{)}

\PY{c+c1}{\PYZsh{} Compose the operator for a Trotter step several times to generate the }
\PY{c+c1}{\PYZsh{} operator for the full time\PYZhy{}evolution}
\PY{c+c1}{\PYZsh{}\PYZsh{}\PYZsh{}INSERT CODE BELOW THIS LINE}

\PY{n}{total\PYZus{}time\PYZus{}evolution\PYZus{}circuit} \PY{o}{=} \PY{n}{PauliTrotterEvolution}\PY{p}{(}\PY{p}{)}\PY{o}{.}\PY{n}{convert}\PY{p}{(}\PY{n}{time\PYZus{}evolution\PYZus{}operator}\PY{p}{)}\PY{o}{.}\PY{n}{assign\PYZus{}parameters}\PY{p}{(}\PY{p}{\PYZob{}}\PY{n}{t}\PY{p}{:}\PY{n}{time\PYZus{}step\PYZus{}value}\PY{p}{\PYZcb{}}\PY{p}{)}\PY{o}{.}\PY{n}{to\PYZus{}circuit\PYZus{}op}\PY{p}{(}\PY{p}{)}

\PY{k}{for} \PY{n}{a} \PY{o+ow}{in} \PY{n+nb}{range}\PY{p}{(}\PY{n}{steps}\PY{o}{\PYZhy{}}\PY{l+m+mi}{1}\PY{p}{)}\PY{p}{:}
     
    \PY{n}{total\PYZus{}time\PYZus{}evolution\PYZus{}circuit} \PY{o}{=} \PY{n}{total\PYZus{}time\PYZus{}evolution\PYZus{}circuit} \PY{o}{@} \PY{n}{PauliTrotterEvolution}\PY{p}{(}\PY{p}{)}\PY{o}{.}\PY{n}{convert}\PY{p}{(}\PY{n}{time\PYZus{}evolution\PYZus{}operator}\PY{p}{)}\PY{o}{.}\PY{n}{assign\PYZus{}parameters}\PY{p}{(}\PY{p}{\PYZob{}}\PY{n}{t}\PY{p}{:}\PY{n}{time\PYZus{}step\PYZus{}value}\PY{p}{\PYZcb{}}\PY{p}{)}\PY{o}{.}\PY{n}{to\PYZus{}circuit\PYZus{}op}\PY{p}{(}\PY{p}{)}
    


\PY{c+c1}{\PYZsh{}\PYZsh{}\PYZsh{}DO NOT EDIT BELOW THIS LINE}

\PY{n+nb}{print}\PY{p}{(}\PY{n}{total\PYZus{}time\PYZus{}evolution\PYZus{}circuit}\PY{p}{)}
\end{Verbatim}
\end{tcolorbox}

    \begin{Verbatim}[commandchars=\\\{\}]
     ┌────────────────────────────────────┐»
q\_0: ┤0                                   ├»
     │  exp(-it (II + XX + YY + ZZ))(0.1) │»
q\_1: ┤1                                   ├»
     └────────────────────────────────────┘»
«     ┌────────────────────────────────────┐»
«q\_0: ┤0                                   ├»
«     │  exp(-it (II + XX + YY + ZZ))(0.1) │»
«q\_1: ┤1                                   ├»
«     └────────────────────────────────────┘»
«     ┌────────────────────────────────────┐»
«q\_0: ┤0                                   ├»
«     │  exp(-it (II + XX + YY + ZZ))(0.1) │»
«q\_1: ┤1                                   ├»
«     └────────────────────────────────────┘»
«     ┌────────────────────────────────────┐»
«q\_0: ┤0                                   ├»
«     │  exp(-it (II + XX + YY + ZZ))(0.1) │»
«q\_1: ┤1                                   ├»
«     └────────────────────────────────────┘»
«     ┌────────────────────────────────────┐
«q\_0: ┤0                                   ├
«     │  exp(-it (II + XX + YY + ZZ))(0.1) │
«q\_1: ┤1                                   ├
«     └────────────────────────────────────┘
    \end{Verbatim}

    \begin{tcolorbox}[breakable, size=fbox, boxrule=1pt, pad at break*=1mm,colback=cellbackground, colframe=cellborder]
\prompt{In}{incolor}{69}{\boxspacing}
\begin{Verbatim}[commandchars=\\\{\}]
\PY{k+kn}{from} \PY{n+nn}{qc\PYZus{}grader}\PY{n+nn}{.}\PY{n+nn}{challenges}\PY{n+nn}{.}\PY{n+nn}{qgss\PYZus{}2022} \PY{k+kn}{import} \PY{n}{grade\PYZus{}lab2\PYZus{}ex11}

\PY{n}{grade\PYZus{}lab2\PYZus{}ex11}\PY{p}{(}\PY{n}{total\PYZus{}time\PYZus{}evolution\PYZus{}circuit}\PY{p}{)}
\end{Verbatim}
\end{tcolorbox}

    \begin{Verbatim}[commandchars=\\\{\}]
Submitting your answer. Please wait{\ldots}
Congratulations 🎉! Your answer is correct and has been submitted.
    \end{Verbatim}

    Consider now a system of three qubits initially in the
\(\vert 001 \rangle\) state, whose evolution is determined by the tight
binding Hamiltonian

\[\hat{H} = \sum_{i=0}^{1} \hat{\sigma}_x^{(i)} \hat{\sigma}_x^{(i+1)}  + \sum_{i=0}^{1} \hat{\sigma}_y^{(i)} \hat{\sigma}_y^{(i+1)} \]

Determine the final state at time \(t=2\) by evolving the initial state
with the time-evolution operator generated by the tight binding
Hamiltonian. You can proceed in a similar way as the previous exercises,
be careful to compose the circuit for the state preparation and for the
time evolution correctly. Note that you should define the Hamiltonian
operator exactly as the definition above to get the right aswer for the
grader.

\hypertarget{construct-the-circuit-for-preparing-an-initial-state-and-evolving-it-under-the-tight-binding-hamiltonian}{%
\subsubsection{8.) Construct the circuit for preparing an initial state
and evolving it under the tight binding
Hamiltonian}\label{construct-the-circuit-for-preparing-an-initial-state-and-evolving-it-under-the-tight-binding-hamiltonian}}

    \begin{tcolorbox}[breakable, size=fbox, boxrule=1pt, pad at break*=1mm,colback=cellbackground, colframe=cellborder]
\prompt{In}{incolor}{70}{\boxspacing}
\begin{Verbatim}[commandchars=\\\{\}]
\PY{c+c1}{\PYZsh{} Set number of qubits}
\PY{n}{num\PYZus{}qubits} \PY{o}{=} \PY{l+m+mi}{3}
\PY{c+c1}{\PYZsh{} Define time parameter}
\PY{n}{t} \PY{o}{=} \PY{n}{Parameter}\PY{p}{(}\PY{l+s+s1}{\PYZsq{}}\PY{l+s+s1}{t}\PY{l+s+s1}{\PYZsq{}}\PY{p}{)}
\PY{c+c1}{\PYZsh{} Set total evolution time}
\PY{n}{evolution\PYZus{}time\PYZus{}t} \PY{o}{=} \PY{l+m+mi}{2}
\PY{c+c1}{\PYZsh{} Set size of time\PYZhy{}step for Trotter evolution}
\PY{n}{time\PYZus{}step\PYZus{}value\PYZus{}t} \PY{o}{=} \PY{l+m+mf}{0.1}
\PY{c+c1}{\PYZsh{} Define the number of steps}
\PY{n}{steps\PYZus{}t} \PY{o}{=} \PY{n+nb}{int}\PY{p}{(}\PY{n}{evolution\PYZus{}time\PYZus{}t}\PY{o}{/}\PY{n}{time\PYZus{}step\PYZus{}value\PYZus{}t}\PY{p}{)}
\PY{c+c1}{\PYZsh{} Create circuit}
\PY{n}{tight\PYZus{}binding\PYZus{}circuit} \PY{o}{=} \PY{n}{QuantumCircuit}\PY{p}{(}\PY{n}{num\PYZus{}qubits}\PY{p}{)}
\PY{c+c1}{\PYZsh{} Add initial state preparation}
\PY{n}{tight\PYZus{}binding\PYZus{}circuit}\PY{o}{.}\PY{n}{x}\PY{p}{(}\PY{l+m+mi}{0}\PY{p}{)}

\PY{n}{full\PYZus{}time\PYZus{}evolution\PYZus{}circuit} \PY{o}{=} \PY{n}{tight\PYZus{}binding\PYZus{}circuit}

\PY{c+c1}{\PYZsh{} Define the Hamiltonian, the time\PYZhy{}evolution operator, the Trotter step and the total evolution}
\PY{c+c1}{\PYZsh{}\PYZsh{}\PYZsh{}INSERT CODE BELOW THIS LINE}

\PY{n}{H1} \PY{o}{=} \PY{p}{(}\PY{n}{I}\PY{o}{\PYZca{}}\PY{n}{X}\PY{o}{\PYZca{}}\PY{n}{X}\PY{p}{)} \PY{o}{+} \PY{p}{(}\PY{n}{X}\PY{o}{\PYZca{}}\PY{n}{X}\PY{o}{\PYZca{}}\PY{n}{I}\PY{p}{)} \PY{o}{+} \PY{p}{(}\PY{n}{I}\PY{o}{\PYZca{}}\PY{n}{Y}\PY{o}{\PYZca{}}\PY{n}{Y}\PY{p}{)} \PY{o}{+} \PY{p}{(}\PY{n}{Y}\PY{o}{\PYZca{}}\PY{n}{Y}\PY{o}{\PYZca{}}\PY{n}{I}\PY{p}{)}

\PY{n}{time\PYZus{}evolution\PYZus{}operator1} \PY{o}{=} \PY{p}{(}\PY{n}{H1}\PY{o}{*}\PY{n}{t}\PY{p}{)}\PY{o}{.}\PY{n}{exp\PYZus{}i}\PY{p}{(}\PY{p}{)}

\PY{n}{full\PYZus{}time\PYZus{}evolution\PYZus{}circuit1} \PY{o}{=} \PY{n}{PauliTrotterEvolution}\PY{p}{(}\PY{p}{)}\PY{o}{.}\PY{n}{convert}\PY{p}{(}\PY{n}{time\PYZus{}evolution\PYZus{}operator1}\PY{p}{)}\PY{o}{.}\PY{n}{assign\PYZus{}parameters}\PY{p}{(}\PY{p}{\PYZob{}}\PY{n}{t}\PY{p}{:}\PY{n}{time\PYZus{}step\PYZus{}value\PYZus{}t}\PY{p}{\PYZcb{}}\PY{p}{)}\PY{o}{.}\PY{n}{to\PYZus{}circuit}\PY{p}{(}\PY{p}{)}

\PY{k}{for} \PY{n}{a} \PY{o+ow}{in} \PY{n+nb}{range}\PY{p}{(}\PY{n}{steps\PYZus{}t}\PY{o}{\PYZhy{}}\PY{l+m+mi}{1}\PY{p}{)}\PY{p}{:}
     
    \PY{n}{full\PYZus{}time\PYZus{}evolution\PYZus{}circuit1} \PY{o}{=} \PY{n}{full\PYZus{}time\PYZus{}evolution\PYZus{}circuit1} \PY{o}{+} \PY{n}{PauliTrotterEvolution}\PY{p}{(}\PY{p}{)}\PY{o}{.}\PY{n}{convert}\PY{p}{(}\PY{n}{time\PYZus{}evolution\PYZus{}operator1}\PY{p}{)}\PY{o}{.}\PY{n}{assign\PYZus{}parameters}\PY{p}{(}\PY{p}{\PYZob{}}\PY{n}{t}\PY{p}{:}\PY{n}{time\PYZus{}step\PYZus{}value\PYZus{}t}\PY{p}{\PYZcb{}}\PY{p}{)}\PY{o}{.}\PY{n}{to\PYZus{}circuit}\PY{p}{(}\PY{p}{)}
    

\PY{n}{full\PYZus{}time\PYZus{}evolution\PYZus{}circuit} \PY{o}{=} \PY{n}{full\PYZus{}time\PYZus{}evolution\PYZus{}circuit} \PY{o}{+} \PY{n}{full\PYZus{}time\PYZus{}evolution\PYZus{}circuit1}
    

\PY{c+c1}{\PYZsh{}\PYZsh{}\PYZsh{}DO NOT EDIT BELOW THIS LINE}

\PY{n+nb}{print}\PY{p}{(}\PY{n}{full\PYZus{}time\PYZus{}evolution\PYZus{}circuit}\PY{p}{)}
\end{Verbatim}
\end{tcolorbox}

    \begin{Verbatim}[commandchars=\\\{\}]
     ┌───┐┌────────────────────────────────────────┐»
q\_0: ┤ X ├┤0                                       ├»
     └───┘│                                        │»
q\_1: ─────┤1 exp(-it (IXX + XXI + IYY + YYI))(0.1) ├»
          │                                        │»
q\_2: ─────┤2                                       ├»
          └────────────────────────────────────────┘»
«     ┌────────────────────────────────────────┐»
«q\_0: ┤0                                       ├»
«     │                                        │»
«q\_1: ┤1 exp(-it (IXX + XXI + IYY + YYI))(0.1) ├»
«     │                                        │»
«q\_2: ┤2                                       ├»
«     └────────────────────────────────────────┘»
«     ┌────────────────────────────────────────┐»
«q\_0: ┤0                                       ├»
«     │                                        │»
«q\_1: ┤1 exp(-it (IXX + XXI + IYY + YYI))(0.1) ├»
«     │                                        │»
«q\_2: ┤2                                       ├»
«     └────────────────────────────────────────┘»
«     ┌────────────────────────────────────────┐»
«q\_0: ┤0                                       ├»
«     │                                        │»
«q\_1: ┤1 exp(-it (IXX + XXI + IYY + YYI))(0.1) ├»
«     │                                        │»
«q\_2: ┤2                                       ├»
«     └────────────────────────────────────────┘»
«     ┌────────────────────────────────────────┐»
«q\_0: ┤0                                       ├»
«     │                                        │»
«q\_1: ┤1 exp(-it (IXX + XXI + IYY + YYI))(0.1) ├»
«     │                                        │»
«q\_2: ┤2                                       ├»
«     └────────────────────────────────────────┘»
«     ┌────────────────────────────────────────┐»
«q\_0: ┤0                                       ├»
«     │                                        │»
«q\_1: ┤1 exp(-it (IXX + XXI + IYY + YYI))(0.1) ├»
«     │                                        │»
«q\_2: ┤2                                       ├»
«     └────────────────────────────────────────┘»
«     ┌────────────────────────────────────────┐»
«q\_0: ┤0                                       ├»
«     │                                        │»
«q\_1: ┤1 exp(-it (IXX + XXI + IYY + YYI))(0.1) ├»
«     │                                        │»
«q\_2: ┤2                                       ├»
«     └────────────────────────────────────────┘»
«     ┌────────────────────────────────────────┐»
«q\_0: ┤0                                       ├»
«     │                                        │»
«q\_1: ┤1 exp(-it (IXX + XXI + IYY + YYI))(0.1) ├»
«     │                                        │»
«q\_2: ┤2                                       ├»
«     └────────────────────────────────────────┘»
«     ┌────────────────────────────────────────┐»
«q\_0: ┤0                                       ├»
«     │                                        │»
«q\_1: ┤1 exp(-it (IXX + XXI + IYY + YYI))(0.1) ├»
«     │                                        │»
«q\_2: ┤2                                       ├»
«     └────────────────────────────────────────┘»
«     ┌────────────────────────────────────────┐»
«q\_0: ┤0                                       ├»
«     │                                        │»
«q\_1: ┤1 exp(-it (IXX + XXI + IYY + YYI))(0.1) ├»
«     │                                        │»
«q\_2: ┤2                                       ├»
«     └────────────────────────────────────────┘»
«     ┌────────────────────────────────────────┐»
«q\_0: ┤0                                       ├»
«     │                                        │»
«q\_1: ┤1 exp(-it (IXX + XXI + IYY + YYI))(0.1) ├»
«     │                                        │»
«q\_2: ┤2                                       ├»
«     └────────────────────────────────────────┘»
«     ┌────────────────────────────────────────┐»
«q\_0: ┤0                                       ├»
«     │                                        │»
«q\_1: ┤1 exp(-it (IXX + XXI + IYY + YYI))(0.1) ├»
«     │                                        │»
«q\_2: ┤2                                       ├»
«     └────────────────────────────────────────┘»
«     ┌────────────────────────────────────────┐»
«q\_0: ┤0                                       ├»
«     │                                        │»
«q\_1: ┤1 exp(-it (IXX + XXI + IYY + YYI))(0.1) ├»
«     │                                        │»
«q\_2: ┤2                                       ├»
«     └────────────────────────────────────────┘»
«     ┌────────────────────────────────────────┐»
«q\_0: ┤0                                       ├»
«     │                                        │»
«q\_1: ┤1 exp(-it (IXX + XXI + IYY + YYI))(0.1) ├»
«     │                                        │»
«q\_2: ┤2                                       ├»
«     └────────────────────────────────────────┘»
«     ┌────────────────────────────────────────┐»
«q\_0: ┤0                                       ├»
«     │                                        │»
«q\_1: ┤1 exp(-it (IXX + XXI + IYY + YYI))(0.1) ├»
«     │                                        │»
«q\_2: ┤2                                       ├»
«     └────────────────────────────────────────┘»
«     ┌────────────────────────────────────────┐»
«q\_0: ┤0                                       ├»
«     │                                        │»
«q\_1: ┤1 exp(-it (IXX + XXI + IYY + YYI))(0.1) ├»
«     │                                        │»
«q\_2: ┤2                                       ├»
«     └────────────────────────────────────────┘»
«     ┌────────────────────────────────────────┐»
«q\_0: ┤0                                       ├»
«     │                                        │»
«q\_1: ┤1 exp(-it (IXX + XXI + IYY + YYI))(0.1) ├»
«     │                                        │»
«q\_2: ┤2                                       ├»
«     └────────────────────────────────────────┘»
«     ┌────────────────────────────────────────┐»
«q\_0: ┤0                                       ├»
«     │                                        │»
«q\_1: ┤1 exp(-it (IXX + XXI + IYY + YYI))(0.1) ├»
«     │                                        │»
«q\_2: ┤2                                       ├»
«     └────────────────────────────────────────┘»
«     ┌────────────────────────────────────────┐»
«q\_0: ┤0                                       ├»
«     │                                        │»
«q\_1: ┤1 exp(-it (IXX + XXI + IYY + YYI))(0.1) ├»
«     │                                        │»
«q\_2: ┤2                                       ├»
«     └────────────────────────────────────────┘»
«     ┌────────────────────────────────────────┐
«q\_0: ┤0                                       ├
«     │                                        │
«q\_1: ┤1 exp(-it (IXX + XXI + IYY + YYI))(0.1) ├
«     │                                        │
«q\_2: ┤2                                       ├
«     └────────────────────────────────────────┘
    \end{Verbatim}

    \begin{tcolorbox}[breakable, size=fbox, boxrule=1pt, pad at break*=1mm,colback=cellbackground, colframe=cellborder]
\prompt{In}{incolor}{71}{\boxspacing}
\begin{Verbatim}[commandchars=\\\{\}]
\PY{k+kn}{from} \PY{n+nn}{qc\PYZus{}grader}\PY{n+nn}{.}\PY{n+nn}{challenges}\PY{n+nn}{.}\PY{n+nn}{qgss\PYZus{}2022} \PY{k+kn}{import} \PY{n}{grade\PYZus{}lab2\PYZus{}ex12}

\PY{n}{grade\PYZus{}lab2\PYZus{}ex12}\PY{p}{(}\PY{n}{full\PYZus{}time\PYZus{}evolution\PYZus{}circuit}\PY{p}{)}
\end{Verbatim}
\end{tcolorbox}

    \begin{Verbatim}[commandchars=\\\{\}]
Submitting your answer. Please wait{\ldots}
Congratulations 🎉! Your answer is correct and has been submitted.
    \end{Verbatim}


    % Add a bibliography block to the postdoc
    
    
    
\end{document}

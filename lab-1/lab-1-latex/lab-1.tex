\documentclass[11pt]{article}

    \usepackage[breakable]{tcolorbox}
    \usepackage{parskip} % Stop auto-indenting (to mimic markdown behaviour)
    
    \usepackage{iftex}
    \ifPDFTeX
    	\usepackage[T1]{fontenc}
    	\usepackage{mathpazo}
    \else
    	\usepackage{fontspec}
    \fi

    % Basic figure setup, for now with no caption control since it's done
    % automatically by Pandoc (which extracts ![](path) syntax from Markdown).
    \usepackage{graphicx}
    % Maintain compatibility with old templates. Remove in nbconvert 6.0
    \let\Oldincludegraphics\includegraphics
    % Ensure that by default, figures have no caption (until we provide a
    % proper Figure object with a Caption API and a way to capture that
    % in the conversion process - todo).
    \usepackage{caption}
    \DeclareCaptionFormat{nocaption}{}
    \captionsetup{format=nocaption,aboveskip=0pt,belowskip=0pt}

    \usepackage[Export]{adjustbox} % Used to constrain images to a maximum size
    \adjustboxset{max size={0.9\linewidth}{0.9\paperheight}}
    \usepackage{float}
    \floatplacement{figure}{H} % forces figures to be placed at the correct location
    \usepackage{xcolor} % Allow colors to be defined
    \usepackage{enumerate} % Needed for markdown enumerations to work
    \usepackage{geometry} % Used to adjust the document margins
    \usepackage{amsmath} % Equations
    \usepackage{amssymb} % Equations
    \usepackage{textcomp} % defines textquotesingle
    % Hack from http://tex.stackexchange.com/a/47451/13684:
    \AtBeginDocument{%
        \def\PYZsq{\textquotesingle}% Upright quotes in Pygmentized code
    }
    \usepackage{upquote} % Upright quotes for verbatim code
    \usepackage{eurosym} % defines \euro
    \usepackage[mathletters]{ucs} % Extended unicode (utf-8) support
    \usepackage{fancyvrb} % verbatim replacement that allows latex
    \usepackage{grffile} % extends the file name processing of package graphics 
                         % to support a larger range
    \makeatletter % fix for grffile with XeLaTeX
    \def\Gread@@xetex#1{%
      \IfFileExists{"\Gin@base".bb}%
      {\Gread@eps{\Gin@base.bb}}%
      {\Gread@@xetex@aux#1}%
    }
    \makeatother

    % The hyperref package gives us a pdf with properly built
    % internal navigation ('pdf bookmarks' for the table of contents,
    % internal cross-reference links, web links for URLs, etc.)
    \usepackage{hyperref}
    % The default LaTeX title has an obnoxious amount of whitespace. By default,
    % titling removes some of it. It also provides customization options.
    \usepackage{titling}
    \usepackage{longtable} % longtable support required by pandoc >1.10
    \usepackage{booktabs}  % table support for pandoc > 1.12.2
    \usepackage[inline]{enumitem} % IRkernel/repr support (it uses the enumerate* environment)
    \usepackage[normalem]{ulem} % ulem is needed to support strikethroughs (\sout)
                                % normalem makes italics be italics, not underlines
    \usepackage{mathrsfs}
    

    
    % Colors for the hyperref package
    \definecolor{urlcolor}{rgb}{0,.145,.698}
    \definecolor{linkcolor}{rgb}{.71,0.21,0.01}
    \definecolor{citecolor}{rgb}{.12,.54,.11}

    % ANSI colors
    \definecolor{ansi-black}{HTML}{3E424D}
    \definecolor{ansi-black-intense}{HTML}{282C36}
    \definecolor{ansi-red}{HTML}{E75C58}
    \definecolor{ansi-red-intense}{HTML}{B22B31}
    \definecolor{ansi-green}{HTML}{00A250}
    \definecolor{ansi-green-intense}{HTML}{007427}
    \definecolor{ansi-yellow}{HTML}{DDB62B}
    \definecolor{ansi-yellow-intense}{HTML}{B27D12}
    \definecolor{ansi-blue}{HTML}{208FFB}
    \definecolor{ansi-blue-intense}{HTML}{0065CA}
    \definecolor{ansi-magenta}{HTML}{D160C4}
    \definecolor{ansi-magenta-intense}{HTML}{A03196}
    \definecolor{ansi-cyan}{HTML}{60C6C8}
    \definecolor{ansi-cyan-intense}{HTML}{258F8F}
    \definecolor{ansi-white}{HTML}{C5C1B4}
    \definecolor{ansi-white-intense}{HTML}{A1A6B2}
    \definecolor{ansi-default-inverse-fg}{HTML}{FFFFFF}
    \definecolor{ansi-default-inverse-bg}{HTML}{000000}

    % commands and environments needed by pandoc snippets
    % extracted from the output of `pandoc -s`
    \providecommand{\tightlist}{%
      \setlength{\itemsep}{0pt}\setlength{\parskip}{0pt}}
    \DefineVerbatimEnvironment{Highlighting}{Verbatim}{commandchars=\\\{\}}
    % Add ',fontsize=\small' for more characters per line
    \newenvironment{Shaded}{}{}
    \newcommand{\KeywordTok}[1]{\textcolor[rgb]{0.00,0.44,0.13}{\textbf{{#1}}}}
    \newcommand{\DataTypeTok}[1]{\textcolor[rgb]{0.56,0.13,0.00}{{#1}}}
    \newcommand{\DecValTok}[1]{\textcolor[rgb]{0.25,0.63,0.44}{{#1}}}
    \newcommand{\BaseNTok}[1]{\textcolor[rgb]{0.25,0.63,0.44}{{#1}}}
    \newcommand{\FloatTok}[1]{\textcolor[rgb]{0.25,0.63,0.44}{{#1}}}
    \newcommand{\CharTok}[1]{\textcolor[rgb]{0.25,0.44,0.63}{{#1}}}
    \newcommand{\StringTok}[1]{\textcolor[rgb]{0.25,0.44,0.63}{{#1}}}
    \newcommand{\CommentTok}[1]{\textcolor[rgb]{0.38,0.63,0.69}{\textit{{#1}}}}
    \newcommand{\OtherTok}[1]{\textcolor[rgb]{0.00,0.44,0.13}{{#1}}}
    \newcommand{\AlertTok}[1]{\textcolor[rgb]{1.00,0.00,0.00}{\textbf{{#1}}}}
    \newcommand{\FunctionTok}[1]{\textcolor[rgb]{0.02,0.16,0.49}{{#1}}}
    \newcommand{\RegionMarkerTok}[1]{{#1}}
    \newcommand{\ErrorTok}[1]{\textcolor[rgb]{1.00,0.00,0.00}{\textbf{{#1}}}}
    \newcommand{\NormalTok}[1]{{#1}}
    
    % Additional commands for more recent versions of Pandoc
    \newcommand{\ConstantTok}[1]{\textcolor[rgb]{0.53,0.00,0.00}{{#1}}}
    \newcommand{\SpecialCharTok}[1]{\textcolor[rgb]{0.25,0.44,0.63}{{#1}}}
    \newcommand{\VerbatimStringTok}[1]{\textcolor[rgb]{0.25,0.44,0.63}{{#1}}}
    \newcommand{\SpecialStringTok}[1]{\textcolor[rgb]{0.73,0.40,0.53}{{#1}}}
    \newcommand{\ImportTok}[1]{{#1}}
    \newcommand{\DocumentationTok}[1]{\textcolor[rgb]{0.73,0.13,0.13}{\textit{{#1}}}}
    \newcommand{\AnnotationTok}[1]{\textcolor[rgb]{0.38,0.63,0.69}{\textbf{\textit{{#1}}}}}
    \newcommand{\CommentVarTok}[1]{\textcolor[rgb]{0.38,0.63,0.69}{\textbf{\textit{{#1}}}}}
    \newcommand{\VariableTok}[1]{\textcolor[rgb]{0.10,0.09,0.49}{{#1}}}
    \newcommand{\ControlFlowTok}[1]{\textcolor[rgb]{0.00,0.44,0.13}{\textbf{{#1}}}}
    \newcommand{\OperatorTok}[1]{\textcolor[rgb]{0.40,0.40,0.40}{{#1}}}
    \newcommand{\BuiltInTok}[1]{{#1}}
    \newcommand{\ExtensionTok}[1]{{#1}}
    \newcommand{\PreprocessorTok}[1]{\textcolor[rgb]{0.74,0.48,0.00}{{#1}}}
    \newcommand{\AttributeTok}[1]{\textcolor[rgb]{0.49,0.56,0.16}{{#1}}}
    \newcommand{\InformationTok}[1]{\textcolor[rgb]{0.38,0.63,0.69}{\textbf{\textit{{#1}}}}}
    \newcommand{\WarningTok}[1]{\textcolor[rgb]{0.38,0.63,0.69}{\textbf{\textit{{#1}}}}}
    
    
    % Define a nice break command that doesn't care if a line doesn't already
    % exist.
    \def\br{\hspace*{\fill} \\* }
    % Math Jax compatibility definitions
    \def\gt{>}
    \def\lt{<}
    \let\Oldtex\TeX
    \let\Oldlatex\LaTeX
    \renewcommand{\TeX}{\textrm{\Oldtex}}
    \renewcommand{\LaTeX}{\textrm{\Oldlatex}}
    % Document parameters
    % Document title
    \title{lab-1}
    
    
    
    
    
% Pygments definitions
\makeatletter
\def\PY@reset{\let\PY@it=\relax \let\PY@bf=\relax%
    \let\PY@ul=\relax \let\PY@tc=\relax%
    \let\PY@bc=\relax \let\PY@ff=\relax}
\def\PY@tok#1{\csname PY@tok@#1\endcsname}
\def\PY@toks#1+{\ifx\relax#1\empty\else%
    \PY@tok{#1}\expandafter\PY@toks\fi}
\def\PY@do#1{\PY@bc{\PY@tc{\PY@ul{%
    \PY@it{\PY@bf{\PY@ff{#1}}}}}}}
\def\PY#1#2{\PY@reset\PY@toks#1+\relax+\PY@do{#2}}

\@namedef{PY@tok@w}{\def\PY@tc##1{\textcolor[rgb]{0.73,0.73,0.73}{##1}}}
\@namedef{PY@tok@c}{\let\PY@it=\textit\def\PY@tc##1{\textcolor[rgb]{0.24,0.48,0.48}{##1}}}
\@namedef{PY@tok@cp}{\def\PY@tc##1{\textcolor[rgb]{0.61,0.40,0.00}{##1}}}
\@namedef{PY@tok@k}{\let\PY@bf=\textbf\def\PY@tc##1{\textcolor[rgb]{0.00,0.50,0.00}{##1}}}
\@namedef{PY@tok@kp}{\def\PY@tc##1{\textcolor[rgb]{0.00,0.50,0.00}{##1}}}
\@namedef{PY@tok@kt}{\def\PY@tc##1{\textcolor[rgb]{0.69,0.00,0.25}{##1}}}
\@namedef{PY@tok@o}{\def\PY@tc##1{\textcolor[rgb]{0.40,0.40,0.40}{##1}}}
\@namedef{PY@tok@ow}{\let\PY@bf=\textbf\def\PY@tc##1{\textcolor[rgb]{0.67,0.13,1.00}{##1}}}
\@namedef{PY@tok@nb}{\def\PY@tc##1{\textcolor[rgb]{0.00,0.50,0.00}{##1}}}
\@namedef{PY@tok@nf}{\def\PY@tc##1{\textcolor[rgb]{0.00,0.00,1.00}{##1}}}
\@namedef{PY@tok@nc}{\let\PY@bf=\textbf\def\PY@tc##1{\textcolor[rgb]{0.00,0.00,1.00}{##1}}}
\@namedef{PY@tok@nn}{\let\PY@bf=\textbf\def\PY@tc##1{\textcolor[rgb]{0.00,0.00,1.00}{##1}}}
\@namedef{PY@tok@ne}{\let\PY@bf=\textbf\def\PY@tc##1{\textcolor[rgb]{0.80,0.25,0.22}{##1}}}
\@namedef{PY@tok@nv}{\def\PY@tc##1{\textcolor[rgb]{0.10,0.09,0.49}{##1}}}
\@namedef{PY@tok@no}{\def\PY@tc##1{\textcolor[rgb]{0.53,0.00,0.00}{##1}}}
\@namedef{PY@tok@nl}{\def\PY@tc##1{\textcolor[rgb]{0.46,0.46,0.00}{##1}}}
\@namedef{PY@tok@ni}{\let\PY@bf=\textbf\def\PY@tc##1{\textcolor[rgb]{0.44,0.44,0.44}{##1}}}
\@namedef{PY@tok@na}{\def\PY@tc##1{\textcolor[rgb]{0.41,0.47,0.13}{##1}}}
\@namedef{PY@tok@nt}{\let\PY@bf=\textbf\def\PY@tc##1{\textcolor[rgb]{0.00,0.50,0.00}{##1}}}
\@namedef{PY@tok@nd}{\def\PY@tc##1{\textcolor[rgb]{0.67,0.13,1.00}{##1}}}
\@namedef{PY@tok@s}{\def\PY@tc##1{\textcolor[rgb]{0.73,0.13,0.13}{##1}}}
\@namedef{PY@tok@sd}{\let\PY@it=\textit\def\PY@tc##1{\textcolor[rgb]{0.73,0.13,0.13}{##1}}}
\@namedef{PY@tok@si}{\let\PY@bf=\textbf\def\PY@tc##1{\textcolor[rgb]{0.64,0.35,0.47}{##1}}}
\@namedef{PY@tok@se}{\let\PY@bf=\textbf\def\PY@tc##1{\textcolor[rgb]{0.67,0.36,0.12}{##1}}}
\@namedef{PY@tok@sr}{\def\PY@tc##1{\textcolor[rgb]{0.64,0.35,0.47}{##1}}}
\@namedef{PY@tok@ss}{\def\PY@tc##1{\textcolor[rgb]{0.10,0.09,0.49}{##1}}}
\@namedef{PY@tok@sx}{\def\PY@tc##1{\textcolor[rgb]{0.00,0.50,0.00}{##1}}}
\@namedef{PY@tok@m}{\def\PY@tc##1{\textcolor[rgb]{0.40,0.40,0.40}{##1}}}
\@namedef{PY@tok@gh}{\let\PY@bf=\textbf\def\PY@tc##1{\textcolor[rgb]{0.00,0.00,0.50}{##1}}}
\@namedef{PY@tok@gu}{\let\PY@bf=\textbf\def\PY@tc##1{\textcolor[rgb]{0.50,0.00,0.50}{##1}}}
\@namedef{PY@tok@gd}{\def\PY@tc##1{\textcolor[rgb]{0.63,0.00,0.00}{##1}}}
\@namedef{PY@tok@gi}{\def\PY@tc##1{\textcolor[rgb]{0.00,0.52,0.00}{##1}}}
\@namedef{PY@tok@gr}{\def\PY@tc##1{\textcolor[rgb]{0.89,0.00,0.00}{##1}}}
\@namedef{PY@tok@ge}{\let\PY@it=\textit}
\@namedef{PY@tok@gs}{\let\PY@bf=\textbf}
\@namedef{PY@tok@gp}{\let\PY@bf=\textbf\def\PY@tc##1{\textcolor[rgb]{0.00,0.00,0.50}{##1}}}
\@namedef{PY@tok@go}{\def\PY@tc##1{\textcolor[rgb]{0.44,0.44,0.44}{##1}}}
\@namedef{PY@tok@gt}{\def\PY@tc##1{\textcolor[rgb]{0.00,0.27,0.87}{##1}}}
\@namedef{PY@tok@err}{\def\PY@bc##1{{\setlength{\fboxsep}{\string -\fboxrule}\fcolorbox[rgb]{1.00,0.00,0.00}{1,1,1}{\strut ##1}}}}
\@namedef{PY@tok@kc}{\let\PY@bf=\textbf\def\PY@tc##1{\textcolor[rgb]{0.00,0.50,0.00}{##1}}}
\@namedef{PY@tok@kd}{\let\PY@bf=\textbf\def\PY@tc##1{\textcolor[rgb]{0.00,0.50,0.00}{##1}}}
\@namedef{PY@tok@kn}{\let\PY@bf=\textbf\def\PY@tc##1{\textcolor[rgb]{0.00,0.50,0.00}{##1}}}
\@namedef{PY@tok@kr}{\let\PY@bf=\textbf\def\PY@tc##1{\textcolor[rgb]{0.00,0.50,0.00}{##1}}}
\@namedef{PY@tok@bp}{\def\PY@tc##1{\textcolor[rgb]{0.00,0.50,0.00}{##1}}}
\@namedef{PY@tok@fm}{\def\PY@tc##1{\textcolor[rgb]{0.00,0.00,1.00}{##1}}}
\@namedef{PY@tok@vc}{\def\PY@tc##1{\textcolor[rgb]{0.10,0.09,0.49}{##1}}}
\@namedef{PY@tok@vg}{\def\PY@tc##1{\textcolor[rgb]{0.10,0.09,0.49}{##1}}}
\@namedef{PY@tok@vi}{\def\PY@tc##1{\textcolor[rgb]{0.10,0.09,0.49}{##1}}}
\@namedef{PY@tok@vm}{\def\PY@tc##1{\textcolor[rgb]{0.10,0.09,0.49}{##1}}}
\@namedef{PY@tok@sa}{\def\PY@tc##1{\textcolor[rgb]{0.73,0.13,0.13}{##1}}}
\@namedef{PY@tok@sb}{\def\PY@tc##1{\textcolor[rgb]{0.73,0.13,0.13}{##1}}}
\@namedef{PY@tok@sc}{\def\PY@tc##1{\textcolor[rgb]{0.73,0.13,0.13}{##1}}}
\@namedef{PY@tok@dl}{\def\PY@tc##1{\textcolor[rgb]{0.73,0.13,0.13}{##1}}}
\@namedef{PY@tok@s2}{\def\PY@tc##1{\textcolor[rgb]{0.73,0.13,0.13}{##1}}}
\@namedef{PY@tok@sh}{\def\PY@tc##1{\textcolor[rgb]{0.73,0.13,0.13}{##1}}}
\@namedef{PY@tok@s1}{\def\PY@tc##1{\textcolor[rgb]{0.73,0.13,0.13}{##1}}}
\@namedef{PY@tok@mb}{\def\PY@tc##1{\textcolor[rgb]{0.40,0.40,0.40}{##1}}}
\@namedef{PY@tok@mf}{\def\PY@tc##1{\textcolor[rgb]{0.40,0.40,0.40}{##1}}}
\@namedef{PY@tok@mh}{\def\PY@tc##1{\textcolor[rgb]{0.40,0.40,0.40}{##1}}}
\@namedef{PY@tok@mi}{\def\PY@tc##1{\textcolor[rgb]{0.40,0.40,0.40}{##1}}}
\@namedef{PY@tok@il}{\def\PY@tc##1{\textcolor[rgb]{0.40,0.40,0.40}{##1}}}
\@namedef{PY@tok@mo}{\def\PY@tc##1{\textcolor[rgb]{0.40,0.40,0.40}{##1}}}
\@namedef{PY@tok@ch}{\let\PY@it=\textit\def\PY@tc##1{\textcolor[rgb]{0.24,0.48,0.48}{##1}}}
\@namedef{PY@tok@cm}{\let\PY@it=\textit\def\PY@tc##1{\textcolor[rgb]{0.24,0.48,0.48}{##1}}}
\@namedef{PY@tok@cpf}{\let\PY@it=\textit\def\PY@tc##1{\textcolor[rgb]{0.24,0.48,0.48}{##1}}}
\@namedef{PY@tok@c1}{\let\PY@it=\textit\def\PY@tc##1{\textcolor[rgb]{0.24,0.48,0.48}{##1}}}
\@namedef{PY@tok@cs}{\let\PY@it=\textit\def\PY@tc##1{\textcolor[rgb]{0.24,0.48,0.48}{##1}}}

\def\PYZbs{\char`\\}
\def\PYZus{\char`\_}
\def\PYZob{\char`\{}
\def\PYZcb{\char`\}}
\def\PYZca{\char`\^}
\def\PYZam{\char`\&}
\def\PYZlt{\char`\<}
\def\PYZgt{\char`\>}
\def\PYZsh{\char`\#}
\def\PYZpc{\char`\%}
\def\PYZdl{\char`\$}
\def\PYZhy{\char`\-}
\def\PYZsq{\char`\'}
\def\PYZdq{\char`\"}
\def\PYZti{\char`\~}
% for compatibility with earlier versions
\def\PYZat{@}
\def\PYZlb{[}
\def\PYZrb{]}
\makeatother


    % For linebreaks inside Verbatim environment from package fancyvrb. 
    \makeatletter
        \newbox\Wrappedcontinuationbox 
        \newbox\Wrappedvisiblespacebox 
        \newcommand*\Wrappedvisiblespace {\textcolor{red}{\textvisiblespace}} 
        \newcommand*\Wrappedcontinuationsymbol {\textcolor{red}{\llap{\tiny$\m@th\hookrightarrow$}}} 
        \newcommand*\Wrappedcontinuationindent {3ex } 
        \newcommand*\Wrappedafterbreak {\kern\Wrappedcontinuationindent\copy\Wrappedcontinuationbox} 
        % Take advantage of the already applied Pygments mark-up to insert 
        % potential linebreaks for TeX processing. 
        %        {, <, #, %, $, ' and ": go to next line. 
        %        _, }, ^, &, >, - and ~: stay at end of broken line. 
        % Use of \textquotesingle for straight quote. 
        \newcommand*\Wrappedbreaksatspecials {% 
            \def\PYGZus{\discretionary{\char`\_}{\Wrappedafterbreak}{\char`\_}}% 
            \def\PYGZob{\discretionary{}{\Wrappedafterbreak\char`\{}{\char`\{}}% 
            \def\PYGZcb{\discretionary{\char`\}}{\Wrappedafterbreak}{\char`\}}}% 
            \def\PYGZca{\discretionary{\char`\^}{\Wrappedafterbreak}{\char`\^}}% 
            \def\PYGZam{\discretionary{\char`\&}{\Wrappedafterbreak}{\char`\&}}% 
            \def\PYGZlt{\discretionary{}{\Wrappedafterbreak\char`\<}{\char`\<}}% 
            \def\PYGZgt{\discretionary{\char`\>}{\Wrappedafterbreak}{\char`\>}}% 
            \def\PYGZsh{\discretionary{}{\Wrappedafterbreak\char`\#}{\char`\#}}% 
            \def\PYGZpc{\discretionary{}{\Wrappedafterbreak\char`\%}{\char`\%}}% 
            \def\PYGZdl{\discretionary{}{\Wrappedafterbreak\char`\$}{\char`\$}}% 
            \def\PYGZhy{\discretionary{\char`\-}{\Wrappedafterbreak}{\char`\-}}% 
            \def\PYGZsq{\discretionary{}{\Wrappedafterbreak\textquotesingle}{\textquotesingle}}% 
            \def\PYGZdq{\discretionary{}{\Wrappedafterbreak\char`\"}{\char`\"}}% 
            \def\PYGZti{\discretionary{\char`\~}{\Wrappedafterbreak}{\char`\~}}% 
        } 
        % Some characters . , ; ? ! / are not pygmentized. 
        % This macro makes them "active" and they will insert potential linebreaks 
        \newcommand*\Wrappedbreaksatpunct {% 
            \lccode`\~`\.\lowercase{\def~}{\discretionary{\hbox{\char`\.}}{\Wrappedafterbreak}{\hbox{\char`\.}}}% 
            \lccode`\~`\,\lowercase{\def~}{\discretionary{\hbox{\char`\,}}{\Wrappedafterbreak}{\hbox{\char`\,}}}% 
            \lccode`\~`\;\lowercase{\def~}{\discretionary{\hbox{\char`\;}}{\Wrappedafterbreak}{\hbox{\char`\;}}}% 
            \lccode`\~`\:\lowercase{\def~}{\discretionary{\hbox{\char`\:}}{\Wrappedafterbreak}{\hbox{\char`\:}}}% 
            \lccode`\~`\?\lowercase{\def~}{\discretionary{\hbox{\char`\?}}{\Wrappedafterbreak}{\hbox{\char`\?}}}% 
            \lccode`\~`\!\lowercase{\def~}{\discretionary{\hbox{\char`\!}}{\Wrappedafterbreak}{\hbox{\char`\!}}}% 
            \lccode`\~`\/\lowercase{\def~}{\discretionary{\hbox{\char`\/}}{\Wrappedafterbreak}{\hbox{\char`\/}}}% 
            \catcode`\.\active
            \catcode`\,\active 
            \catcode`\;\active
            \catcode`\:\active
            \catcode`\?\active
            \catcode`\!\active
            \catcode`\/\active 
            \lccode`\~`\~ 	
        }
    \makeatother

    \let\OriginalVerbatim=\Verbatim
    \makeatletter
    \renewcommand{\Verbatim}[1][1]{%
        %\parskip\z@skip
        \sbox\Wrappedcontinuationbox {\Wrappedcontinuationsymbol}%
        \sbox\Wrappedvisiblespacebox {\FV@SetupFont\Wrappedvisiblespace}%
        \def\FancyVerbFormatLine ##1{\hsize\linewidth
            \vtop{\raggedright\hyphenpenalty\z@\exhyphenpenalty\z@
                \doublehyphendemerits\z@\finalhyphendemerits\z@
                \strut ##1\strut}%
        }%
        % If the linebreak is at a space, the latter will be displayed as visible
        % space at end of first line, and a continuation symbol starts next line.
        % Stretch/shrink are however usually zero for typewriter font.
        \def\FV@Space {%
            \nobreak\hskip\z@ plus\fontdimen3\font minus\fontdimen4\font
            \discretionary{\copy\Wrappedvisiblespacebox}{\Wrappedafterbreak}
            {\kern\fontdimen2\font}%
        }%
        
        % Allow breaks at special characters using \PYG... macros.
        \Wrappedbreaksatspecials
        % Breaks at punctuation characters . , ; ? ! and / need catcode=\active 	
        \OriginalVerbatim[#1,codes*=\Wrappedbreaksatpunct]%
    }
    \makeatother

    % Exact colors from NB
    \definecolor{incolor}{HTML}{303F9F}
    \definecolor{outcolor}{HTML}{D84315}
    \definecolor{cellborder}{HTML}{CFCFCF}
    \definecolor{cellbackground}{HTML}{F7F7F7}
    
    % prompt
    \makeatletter
    \newcommand{\boxspacing}{\kern\kvtcb@left@rule\kern\kvtcb@boxsep}
    \makeatother
    \newcommand{\prompt}[4]{
        \ttfamily\llap{{\color{#2}[#3]:\hspace{3pt}#4}}\vspace{-\baselineskip}
    }
    

    
    % Prevent overflowing lines due to hard-to-break entities
    \sloppy 
    % Setup hyperref package
    \hypersetup{
      breaklinks=true,  % so long urls are correctly broken across lines
      colorlinks=true,
      urlcolor=urlcolor,
      linkcolor=linkcolor,
      citecolor=citecolor,
      }
    % Slightly bigger margins than the latex defaults
    
    \geometry{verbose,tmargin=1in,bmargin=1in,lmargin=1in,rmargin=1in}
    
    

\begin{document}
    
    \maketitle
    
    

    
    \hypertarget{part-i-introduction-to-qiskit}{%
\section{Part I: Introduction to
Qiskit}\label{part-i-introduction-to-qiskit}}

Welcome to Qiskit! Before starting with the exercises, please run the
cell below by pressing `shift' + `return'. You can run the other
following cells in the same way.

    \begin{tcolorbox}[breakable, size=fbox, boxrule=1pt, pad at break*=1mm,colback=cellbackground, colframe=cellborder]
\prompt{In}{incolor}{65}{\boxspacing}
\begin{Verbatim}[commandchars=\\\{\}]
\PY{k+kn}{import} \PY{n+nn}{numpy} \PY{k}{as} \PY{n+nn}{np}
\PY{k+kn}{from} \PY{n+nn}{numpy} \PY{k+kn}{import} \PY{n}{pi}

\PY{c+c1}{\PYZsh{} Importing standard Qiskit libraries}
\PY{k+kn}{from} \PY{n+nn}{qiskit} \PY{k+kn}{import} \PY{n}{QuantumCircuit}\PY{p}{,} \PY{n}{transpile}\PY{p}{,} \PY{n}{assemble}\PY{p}{,} \PY{n}{Aer}\PY{p}{,} \PY{n}{IBMQ}\PY{p}{,} \PY{n}{execute}
\PY{k+kn}{from} \PY{n+nn}{qiskit}\PY{n+nn}{.}\PY{n+nn}{quantum\PYZus{}info} \PY{k+kn}{import} \PY{n}{Statevector}
\PY{k+kn}{from} \PY{n+nn}{qiskit}\PY{n+nn}{.}\PY{n+nn}{visualization} \PY{k+kn}{import} \PY{n}{plot\PYZus{}bloch\PYZus{}multivector}\PY{p}{,} \PY{n}{plot\PYZus{}histogram}
\PY{k+kn}{from} \PY{n+nn}{qiskit\PYZus{}textbook}\PY{n+nn}{.}\PY{n+nn}{problems} \PY{k+kn}{import} \PY{n}{dj\PYZus{}problem\PYZus{}oracle}
\end{Verbatim}
\end{tcolorbox}

    \hypertarget{i.1-basic-rotations-on-one-qubit-and-measurements-on-the-bloch-sphere}{%
\subsection{I.1: Basic Rotations on One Qubit and Measurements on the
Bloch
Sphere}\label{i.1-basic-rotations-on-one-qubit-and-measurements-on-the-bloch-sphere}}

Before getting into complicated circuits on many qubits, let us start by
looking at a single qubit. Read this chapter:
https://qiskit.org/textbook/ch-states/introduction.html

It will help you to learn the basics about the Bloch sphere, Pauli
operators, as well as the Hadamard gate and the \(S\) and \(S^\dagger\)
gates.

By default, states in Qiskit start in \(|0\rangle\), which corresponds
to ``arrow up'' on the Bloch sphere. Play around with the gates \(X\),
\(Y\), \(Z\), \(H\), \(S\) and \(S^\dagger\) to get a feeling for the
different rotations. To do so, insert combinations of the following code
lines in the lines indicated in the program:

\begin{verbatim}
qc.x(0)    # rotation by Pi around the x-axis
qc.y(0)    # rotation by Pi around the y-axis
qc.z(0)    # rotation by Pi around the z-axis
qc.s(0)    # rotation by Pi/2 around the z-axis
qc.sdg(0)  # rotation by -Pi/2 around the z-axis
qc.h(0)    # rotation by Pi around an axis located halfway between x and z
\end{verbatim}

Try to reach the given state in the Bloch sphere in each of the
following exercises by applying the correct rotations. (Press Shift +
Enter to run a code cell) \#\#\# 1.) Let us start easy by performing a
bit flip. The goal is to reach the state \(|1\rangle\).

    \begin{tcolorbox}[breakable, size=fbox, boxrule=1pt, pad at break*=1mm,colback=cellbackground, colframe=cellborder]
\prompt{In}{incolor}{66}{\boxspacing}
\begin{Verbatim}[commandchars=\\\{\}]
\PY{k}{def} \PY{n+nf}{lab1\PYZus{}ex1}\PY{p}{(}\PY{p}{)}\PY{p}{:}
    \PY{n}{qc} \PY{o}{=} \PY{n}{QuantumCircuit}\PY{p}{(}\PY{l+m+mi}{1}\PY{p}{)}
    \PY{n}{qc}\PY{o}{.}\PY{n}{x}\PY{p}{(}\PY{l+m+mi}{0}\PY{p}{)}    \PY{c+c1}{\PYZsh{} rotation by Pi around the x\PYZhy{}axis}
    \PY{k}{return} \PY{n}{qc}


\PY{n}{state} \PY{o}{=} \PY{n}{Statevector}\PY{o}{.}\PY{n}{from\PYZus{}instruction}\PY{p}{(}\PY{n}{lab1\PYZus{}ex1}\PY{p}{(}\PY{p}{)}\PY{p}{)}
\PY{n}{plot\PYZus{}bloch\PYZus{}multivector}\PY{p}{(}\PY{n}{state}\PY{p}{)}
\end{Verbatim}
\end{tcolorbox}
 
            
\prompt{Out}{outcolor}{66}{}
    
    \begin{center}
    \adjustimage{max size={0.9\linewidth}{0.9\paperheight}}{output_3_0.png}
    \end{center}
    { \hspace*{\fill} \\}
    

    \begin{tcolorbox}[breakable, size=fbox, boxrule=1pt, pad at break*=1mm,colback=cellbackground, colframe=cellborder]
\prompt{In}{incolor}{67}{\boxspacing}
\begin{Verbatim}[commandchars=\\\{\}]
\PY{k+kn}{from} \PY{n+nn}{qc\PYZus{}grader}\PY{n+nn}{.}\PY{n+nn}{challenges}\PY{n+nn}{.}\PY{n+nn}{qgss\PYZus{}2022} \PY{k+kn}{import} \PY{n}{grade\PYZus{}lab1\PYZus{}ex1}

\PY{c+c1}{\PYZsh{} Note that the grading function is expecting a quantum circuit without measurements}
\PY{n}{grade\PYZus{}lab1\PYZus{}ex1}\PY{p}{(}\PY{n}{lab1\PYZus{}ex1}\PY{p}{(}\PY{p}{)}\PY{p}{)}
\end{Verbatim}
\end{tcolorbox}

    \begin{Verbatim}[commandchars=\\\{\}]
Submitting your answer. Please wait{\ldots}
Congratulations 🎉! Your answer is correct and has been submitted.
    \end{Verbatim}

    \hypertarget{next-we-would-like-to-create-superposition.-the-goal-is-to-reach-the-state-rangle-frac1sqrt2left0rangle-1rangleright.}{%
\subsubsection{\texorpdfstring{2.) Next, we would like to create
superposition. The goal is to reach the state
\(|+\rangle = \frac{1}{\sqrt{2}}\left(|0\rangle + |1\rangle\right)\).}{2.) Next, we would like to create superposition. The goal is to reach the state \textbar+\textbackslash rangle = \textbackslash frac\{1\}\{\textbackslash sqrt\{2\}\}\textbackslash left(\textbar0\textbackslash rangle + \textbar1\textbackslash rangle\textbackslash right).}}\label{next-we-would-like-to-create-superposition.-the-goal-is-to-reach-the-state-rangle-frac1sqrt2left0rangle-1rangleright.}}

    \begin{tcolorbox}[breakable, size=fbox, boxrule=1pt, pad at break*=1mm,colback=cellbackground, colframe=cellborder]
\prompt{In}{incolor}{68}{\boxspacing}
\begin{Verbatim}[commandchars=\\\{\}]
\PY{k}{def} \PY{n+nf}{lab1\PYZus{}ex2}\PY{p}{(}\PY{p}{)}\PY{p}{:}
    \PY{n}{qc} \PY{o}{=} \PY{n}{QuantumCircuit}\PY{p}{(}\PY{l+m+mi}{1}\PY{p}{)}
    \PY{n}{qc}\PY{o}{.}\PY{n}{h}\PY{p}{(}\PY{l+m+mi}{0}\PY{p}{)}
    \PY{k}{return} \PY{n}{qc}

\PY{n}{state} \PY{o}{=} \PY{n}{Statevector}\PY{o}{.}\PY{n}{from\PYZus{}instruction}\PY{p}{(}\PY{n}{lab1\PYZus{}ex2}\PY{p}{(}\PY{p}{)}\PY{p}{)}
\PY{n}{plot\PYZus{}bloch\PYZus{}multivector}\PY{p}{(}\PY{n}{state}\PY{p}{)}
\end{Verbatim}
\end{tcolorbox}
 
            
\prompt{Out}{outcolor}{68}{}
    
    \begin{center}
    \adjustimage{max size={0.9\linewidth}{0.9\paperheight}}{output_6_0.png}
    \end{center}
    { \hspace*{\fill} \\}
    

    \begin{tcolorbox}[breakable, size=fbox, boxrule=1pt, pad at break*=1mm,colback=cellbackground, colframe=cellborder]
\prompt{In}{incolor}{69}{\boxspacing}
\begin{Verbatim}[commandchars=\\\{\}]
\PY{k+kn}{from} \PY{n+nn}{qc\PYZus{}grader}\PY{n+nn}{.}\PY{n+nn}{challenges}\PY{n+nn}{.}\PY{n+nn}{qgss\PYZus{}2022} \PY{k+kn}{import} \PY{n}{grade\PYZus{}lab1\PYZus{}ex2}

\PY{c+c1}{\PYZsh{} Note that the grading function is expecting a quantum circuit without measurements}
\PY{n}{grade\PYZus{}lab1\PYZus{}ex2}\PY{p}{(}\PY{n}{lab1\PYZus{}ex2}\PY{p}{(}\PY{p}{)}\PY{p}{)}
\end{Verbatim}
\end{tcolorbox}

    \begin{Verbatim}[commandchars=\\\{\}]
Submitting your answer. Please wait{\ldots}
Congratulations 🎉! Your answer is correct and has been submitted.
    \end{Verbatim}

    \hypertarget{lets-combine-the-two-operations-seen-before.-the-goal-is-to-reach-the-state--rangle-frac1sqrt2left0rangle---1rangleright.}{%
\subsubsection{\texorpdfstring{3.) Let's combine the two operations seen
before. The goal is to reach the state
\(|-\rangle = \frac{1}{\sqrt{2}}\left(|0\rangle - |1\rangle\right)\).}{3.) Let's combine the two operations seen before. The goal is to reach the state \textbar-\textbackslash rangle = \textbackslash frac\{1\}\{\textbackslash sqrt\{2\}\}\textbackslash left(\textbar0\textbackslash rangle - \textbar1\textbackslash rangle\textbackslash right).}}\label{lets-combine-the-two-operations-seen-before.-the-goal-is-to-reach-the-state--rangle-frac1sqrt2left0rangle---1rangleright.}}

Can you even come up with different ways?

    \begin{tcolorbox}[breakable, size=fbox, boxrule=1pt, pad at break*=1mm,colback=cellbackground, colframe=cellborder]
\prompt{In}{incolor}{70}{\boxspacing}
\begin{Verbatim}[commandchars=\\\{\}]
\PY{k}{def} \PY{n+nf}{lab1\PYZus{}ex3}\PY{p}{(}\PY{p}{)}\PY{p}{:}
    \PY{n}{qc} \PY{o}{=} \PY{n}{QuantumCircuit}\PY{p}{(}\PY{l+m+mi}{1}\PY{p}{)}
    \PY{n}{qc}\PY{o}{.}\PY{n}{x}\PY{p}{(}\PY{l+m+mi}{0}\PY{p}{)}
    \PY{n}{qc}\PY{o}{.}\PY{n}{h}\PY{p}{(}\PY{l+m+mi}{0}\PY{p}{)}
    \PY{k}{return} \PY{n}{qc}

\PY{n}{state} \PY{o}{=} \PY{n}{Statevector}\PY{o}{.}\PY{n}{from\PYZus{}instruction}\PY{p}{(}\PY{n}{lab1\PYZus{}ex3}\PY{p}{(}\PY{p}{)}\PY{p}{)}
\PY{n}{plot\PYZus{}bloch\PYZus{}multivector}\PY{p}{(}\PY{n}{state}\PY{p}{)}
\end{Verbatim}
\end{tcolorbox}
 
            
\prompt{Out}{outcolor}{70}{}
    
    \begin{center}
    \adjustimage{max size={0.9\linewidth}{0.9\paperheight}}{output_9_0.png}
    \end{center}
    { \hspace*{\fill} \\}
    

    \begin{tcolorbox}[breakable, size=fbox, boxrule=1pt, pad at break*=1mm,colback=cellbackground, colframe=cellborder]
\prompt{In}{incolor}{71}{\boxspacing}
\begin{Verbatim}[commandchars=\\\{\}]
\PY{k+kn}{from} \PY{n+nn}{qc\PYZus{}grader}\PY{n+nn}{.}\PY{n+nn}{challenges}\PY{n+nn}{.}\PY{n+nn}{qgss\PYZus{}2022} \PY{k+kn}{import} \PY{n}{grade\PYZus{}lab1\PYZus{}ex3}

\PY{c+c1}{\PYZsh{} Note that the grading function is expecting a quantum circuit without measurements}
\PY{n}{grade\PYZus{}lab1\PYZus{}ex3}\PY{p}{(}\PY{n}{lab1\PYZus{}ex3}\PY{p}{(}\PY{p}{)}\PY{p}{)}
\end{Verbatim}
\end{tcolorbox}

    \begin{Verbatim}[commandchars=\\\{\}]
Submitting your answer. Please wait{\ldots}
Congratulations 🎉! Your answer is correct and has been submitted.
    \end{Verbatim}

    \hypertarget{finally-we-move-on-to-the-complex-numbers.-the-goal-is-to-reach-the-state---irangle-frac1sqrt2left0rangle---i1rangleright.}{%
\subsubsection{\texorpdfstring{4.) Finally, we move on to the complex
numbers. The goal is to reach the state
\(|- i\rangle = \frac{1}{\sqrt{2}}\left(|0\rangle - i|1\rangle\right)\).}{4.) Finally, we move on to the complex numbers. The goal is to reach the state \textbar- i\textbackslash rangle = \textbackslash frac\{1\}\{\textbackslash sqrt\{2\}\}\textbackslash left(\textbar0\textbackslash rangle - i\textbar1\textbackslash rangle\textbackslash right).}}\label{finally-we-move-on-to-the-complex-numbers.-the-goal-is-to-reach-the-state---irangle-frac1sqrt2left0rangle---i1rangleright.}}

    \begin{tcolorbox}[breakable, size=fbox, boxrule=1pt, pad at break*=1mm,colback=cellbackground, colframe=cellborder]
\prompt{In}{incolor}{72}{\boxspacing}
\begin{Verbatim}[commandchars=\\\{\}]
\PY{k}{def} \PY{n+nf}{lab1\PYZus{}ex4}\PY{p}{(}\PY{p}{)}\PY{p}{:}
    \PY{n}{qc} \PY{o}{=} \PY{n}{QuantumCircuit}\PY{p}{(}\PY{l+m+mi}{1}\PY{p}{)}
    \PY{n}{qc}\PY{o}{.}\PY{n}{h}\PY{p}{(}\PY{l+m+mi}{0}\PY{p}{)}
    \PY{n}{qc}\PY{o}{.}\PY{n}{s}\PY{p}{(}\PY{l+m+mi}{0}\PY{p}{)}
    \PY{n}{qc}\PY{o}{.}\PY{n}{h}\PY{p}{(}\PY{l+m+mi}{0}\PY{p}{)}
    \PY{k}{return} \PY{n}{qc}

\PY{n}{state} \PY{o}{=} \PY{n}{Statevector}\PY{o}{.}\PY{n}{from\PYZus{}instruction}\PY{p}{(}\PY{n}{lab1\PYZus{}ex4}\PY{p}{(}\PY{p}{)}\PY{p}{)}
\PY{n}{plot\PYZus{}bloch\PYZus{}multivector}\PY{p}{(}\PY{n}{state}\PY{p}{)}
\end{Verbatim}
\end{tcolorbox}
 
            
\prompt{Out}{outcolor}{72}{}
    
    \begin{center}
    \adjustimage{max size={0.9\linewidth}{0.9\paperheight}}{output_12_0.png}
    \end{center}
    { \hspace*{\fill} \\}
    

    \begin{tcolorbox}[breakable, size=fbox, boxrule=1pt, pad at break*=1mm,colback=cellbackground, colframe=cellborder]
\prompt{In}{incolor}{73}{\boxspacing}
\begin{Verbatim}[commandchars=\\\{\}]
\PY{k+kn}{from} \PY{n+nn}{qc\PYZus{}grader}\PY{n+nn}{.}\PY{n+nn}{challenges}\PY{n+nn}{.}\PY{n+nn}{qgss\PYZus{}2022} \PY{k+kn}{import} \PY{n}{grade\PYZus{}lab1\PYZus{}ex4}

\PY{c+c1}{\PYZsh{} Note that the grading function is expecting a quantum circuit without measurements}
\PY{n}{grade\PYZus{}lab1\PYZus{}ex4}\PY{p}{(}\PY{n}{lab1\PYZus{}ex4}\PY{p}{(}\PY{p}{)}\PY{p}{)}
\end{Verbatim}
\end{tcolorbox}

    \begin{Verbatim}[commandchars=\\\{\}]
Submitting your answer. Please wait{\ldots}
Congratulations 🎉! Your answer is correct and has been submitted.
    \end{Verbatim}

    \hypertarget{i.2-quantum-circuits-using-multi-qubit-gates}{%
\subsection{I.2: Quantum Circuits Using Multi-Qubit
Gates}\label{i.2-quantum-circuits-using-multi-qubit-gates}}

Great job! Now that you've understood the single-qubit gates, let us
look at gates on multiple qubits. Check out this chapter if you would
like to refresh the theory:
https://qiskit.org/textbook/ch-gates/introduction.html

The basic gates on two and three qubits are given by

\begin{verbatim}
qc.cx(c,t)       # controlled-X (= CNOT) gate with control qubit c and target qubit t
qc.cz(c,t)       # controlled-Z gate with control qubit c and target qubit t
qc.ccx(c1,c2,t)  # controlled-controlled-X (= Toffoli) gate with control qubits c1 and c2 and target qubit t
qc.swap(a,b)     # SWAP gate that swaps the states of qubit a and qubit b
\end{verbatim}

We start with an easy gate on two qubits, the controlled-NOT (also CNOT)
gate. The CNOT gate has no effect when applied on two qubits in state
\(|0\rangle\), but this changes if we apply a Hadamard gate before to
the control qubit to bring it in superposition. This way, we can create
entanglement. The resulting state is one of the so-called Bell states.
There are four Bell states in total, so let's try to also construct
another one:

\hypertarget{construct-the-bell-state-psirangle-frac1sqrt2left01rangle-10rangleright.}{%
\subsubsection{\texorpdfstring{5.) Construct the Bell state
\(|\Psi^+\rangle = \frac{1}{\sqrt{2}}\left(|01\rangle + |10\rangle\right)\).}{5.) Construct the Bell state \textbar\textbackslash Psi\^{}+\textbackslash rangle = \textbackslash frac\{1\}\{\textbackslash sqrt\{2\}\}\textbackslash left(\textbar01\textbackslash rangle + \textbar10\textbackslash rangle\textbackslash right).}}\label{construct-the-bell-state-psirangle-frac1sqrt2left01rangle-10rangleright.}}

    \begin{tcolorbox}[breakable, size=fbox, boxrule=1pt, pad at break*=1mm,colback=cellbackground, colframe=cellborder]
\prompt{In}{incolor}{74}{\boxspacing}
\begin{Verbatim}[commandchars=\\\{\}]
\PY{k}{def} \PY{n+nf}{lab1\PYZus{}ex5}\PY{p}{(}\PY{p}{)}\PY{p}{:}
    \PY{c+c1}{\PYZsh{} This time, we not only want two qubits, but also two classical bits for the measurement}
    \PY{n}{qc} \PY{o}{=} \PY{n}{QuantumCircuit}\PY{p}{(}\PY{l+m+mi}{2}\PY{p}{,}\PY{l+m+mi}{2}\PY{p}{)} 
    \PY{n}{qc}\PY{o}{.}\PY{n}{h}\PY{p}{(}\PY{l+m+mi}{0}\PY{p}{)}
    \PY{n}{qc}\PY{o}{.}\PY{n}{cx}\PY{p}{(}\PY{l+m+mi}{0}\PY{p}{,}\PY{l+m+mi}{1}\PY{p}{)}
    \PY{n}{qc}\PY{o}{.}\PY{n}{x}\PY{p}{(}\PY{l+m+mi}{0}\PY{p}{)}
    \PY{k}{return} \PY{n}{qc}

\PY{n}{qc} \PY{o}{=} \PY{n}{lab1\PYZus{}ex5}\PY{p}{(}\PY{p}{)}
\PY{n}{qc}\PY{o}{.}\PY{n}{draw}\PY{p}{(}\PY{p}{)} \PY{c+c1}{\PYZsh{} we draw the circuit}
\end{Verbatim}
\end{tcolorbox}
 
            
\prompt{Out}{outcolor}{74}{}
    
    \begin{center}
    \adjustimage{max size={0.9\linewidth}{0.9\paperheight}}{output_15_0.png}
    \end{center}
    { \hspace*{\fill} \\}
    

    \begin{tcolorbox}[breakable, size=fbox, boxrule=1pt, pad at break*=1mm,colback=cellbackground, colframe=cellborder]
\prompt{In}{incolor}{75}{\boxspacing}
\begin{Verbatim}[commandchars=\\\{\}]
\PY{k+kn}{from} \PY{n+nn}{qc\PYZus{}grader}\PY{n+nn}{.}\PY{n+nn}{challenges}\PY{n+nn}{.}\PY{n+nn}{qgss\PYZus{}2022} \PY{k+kn}{import} \PY{n}{grade\PYZus{}lab1\PYZus{}ex5}

\PY{c+c1}{\PYZsh{} Note that the grading function is expecting a quantum circuit without measurements}
\PY{n}{grade\PYZus{}lab1\PYZus{}ex5}\PY{p}{(}\PY{n}{lab1\PYZus{}ex5}\PY{p}{(}\PY{p}{)}\PY{p}{)}
\end{Verbatim}
\end{tcolorbox}

    \begin{Verbatim}[commandchars=\\\{\}]
Submitting your answer. Please wait{\ldots}
Congratulations 🎉! Your answer is correct and has been submitted.
    \end{Verbatim}

    Let us now also add a measurement to the above circuit so that we can
execute it (using the simulator) and plot the histogram of the
corresponding counts.

    \begin{tcolorbox}[breakable, size=fbox, boxrule=1pt, pad at break*=1mm,colback=cellbackground, colframe=cellborder]
\prompt{In}{incolor}{76}{\boxspacing}
\begin{Verbatim}[commandchars=\\\{\}]
\PY{n}{qc}\PY{o}{.}\PY{n}{measure\PYZus{}all}\PY{p}{(}\PY{p}{)} \PY{c+c1}{\PYZsh{} we measure all the qubits}
\PY{n}{backend} \PY{o}{=} \PY{n}{Aer}\PY{o}{.}\PY{n}{get\PYZus{}backend}\PY{p}{(}\PY{l+s+s1}{\PYZsq{}}\PY{l+s+s1}{qasm\PYZus{}simulator}\PY{l+s+s1}{\PYZsq{}}\PY{p}{)} \PY{c+c1}{\PYZsh{} we choose the simulator as our backend}
\PY{n}{counts} \PY{o}{=} \PY{n}{execute}\PY{p}{(}\PY{n}{qc}\PY{p}{,} \PY{n}{backend}\PY{p}{,} \PY{n}{shots} \PY{o}{=} \PY{l+m+mi}{1000}\PY{p}{)}\PY{o}{.}\PY{n}{result}\PY{p}{(}\PY{p}{)}\PY{o}{.}\PY{n}{get\PYZus{}counts}\PY{p}{(}\PY{p}{)} \PY{c+c1}{\PYZsh{} we run the simulation and get the counts}
\PY{n}{plot\PYZus{}histogram}\PY{p}{(}\PY{n}{counts}\PY{p}{)} \PY{c+c1}{\PYZsh{} let us plot a histogram to see the possible outcomes and corresponding probabilities}
\end{Verbatim}
\end{tcolorbox}
 
            
\prompt{Out}{outcolor}{76}{}
    
    \begin{center}
    \adjustimage{max size={0.9\linewidth}{0.9\paperheight}}{output_18_0.png}
    \end{center}
    { \hspace*{\fill} \\}
    

    As you can see in the histogram, the only possible outputs are ``01''
and ``10'', so the states of the two qubits are always perfectly
anti-correlated.

    \hypertarget{write-a-function-that-builds-a-quantum-circuit-on-3-qubits-and-creates-the-ghz-like-state-psirangle-frac1sqrt2-left011rangle---100-rangle-right.}{%
\subsubsection{\texorpdfstring{6.) Write a function that builds a
quantum circuit on 3 qubits and creates the GHZ-like state,
\(|\Psi\rangle = \frac{1}{\sqrt{2}} \left(|011\rangle - |100 \rangle \right)\).}{6.) Write a function that builds a quantum circuit on 3 qubits and creates the GHZ-like state, \textbar\textbackslash Psi\textbackslash rangle = \textbackslash frac\{1\}\{\textbackslash sqrt\{2\}\} \textbackslash left(\textbar011\textbackslash rangle - \textbar100 \textbackslash rangle \textbackslash right).}}\label{write-a-function-that-builds-a-quantum-circuit-on-3-qubits-and-creates-the-ghz-like-state-psirangle-frac1sqrt2-left011rangle---100-rangle-right.}}

Hint: the following circuit constructs the GHZ state,
\(|GHZ\rangle = \frac{1}{\sqrt{2}} \left(|000\rangle + |111 \rangle \right)\):

    \begin{tcolorbox}[breakable, size=fbox, boxrule=1pt, pad at break*=1mm,colback=cellbackground, colframe=cellborder]
\prompt{In}{incolor}{77}{\boxspacing}
\begin{Verbatim}[commandchars=\\\{\}]
\PY{k}{def} \PY{n+nf}{lab1\PYZus{}ex6}\PY{p}{(}\PY{p}{)}\PY{p}{:}
    \PY{c+c1}{\PYZsh{} This time, we need 3 qubits and also add 3 classical bits in case we want to measure}
    \PY{n}{qc} \PY{o}{=} \PY{n}{QuantumCircuit}\PY{p}{(}\PY{l+m+mi}{3}\PY{p}{,}\PY{l+m+mi}{3}\PY{p}{)} 
    \PY{n}{qc}\PY{o}{.}\PY{n}{x}\PY{p}{(}\PY{l+m+mi}{0}\PY{p}{)}
    \PY{n}{qc}\PY{o}{.}\PY{n}{h}\PY{p}{(}\PY{l+m+mi}{0}\PY{p}{)}
    \PY{n}{qc}\PY{o}{.}\PY{n}{cx}\PY{p}{(}\PY{l+m+mi}{0}\PY{p}{,}\PY{l+m+mi}{1}\PY{p}{)}
    \PY{n}{qc}\PY{o}{.}\PY{n}{x}\PY{p}{(}\PY{l+m+mi}{0}\PY{p}{)}
    \PY{n}{qc}\PY{o}{.}\PY{n}{x}\PY{p}{(}\PY{l+m+mi}{1}\PY{p}{)}
    \PY{n}{qc}\PY{o}{.}\PY{n}{cx}\PY{p}{(}\PY{l+m+mi}{1}\PY{p}{,}\PY{l+m+mi}{2}\PY{p}{)}
    \PY{n}{qc}\PY{o}{.}\PY{n}{x}\PY{p}{(}\PY{l+m+mi}{2}\PY{p}{)}
    \PY{k}{return} \PY{n}{qc}

\PY{n}{qc} \PY{o}{=} \PY{n}{lab1\PYZus{}ex6}\PY{p}{(}\PY{p}{)}
\PY{n}{qc}\PY{o}{.}\PY{n}{draw}\PY{p}{(}\PY{p}{)} \PY{c+c1}{\PYZsh{} we draw the circuit}
\end{Verbatim}
\end{tcolorbox}
 
            
\prompt{Out}{outcolor}{77}{}
    
    \begin{center}
    \adjustimage{max size={0.9\linewidth}{0.9\paperheight}}{output_21_0.png}
    \end{center}
    { \hspace*{\fill} \\}
    

    We can now also measure this circuit the same way we did before.

    \begin{tcolorbox}[breakable, size=fbox, boxrule=1pt, pad at break*=1mm,colback=cellbackground, colframe=cellborder]
\prompt{In}{incolor}{78}{\boxspacing}
\begin{Verbatim}[commandchars=\\\{\}]
\PY{n}{qc}\PY{o}{.}\PY{n}{measure\PYZus{}all}\PY{p}{(}\PY{p}{)} \PY{c+c1}{\PYZsh{} we measure all the qubits}
\PY{n}{backend} \PY{o}{=} \PY{n}{Aer}\PY{o}{.}\PY{n}{get\PYZus{}backend}\PY{p}{(}\PY{l+s+s1}{\PYZsq{}}\PY{l+s+s1}{qasm\PYZus{}simulator}\PY{l+s+s1}{\PYZsq{}}\PY{p}{)} \PY{c+c1}{\PYZsh{} we choose the simulator as our backend}
\PY{n}{counts} \PY{o}{=} \PY{n}{execute}\PY{p}{(}\PY{n}{qc}\PY{p}{,} \PY{n}{backend}\PY{p}{,} \PY{n}{shots} \PY{o}{=} \PY{l+m+mi}{1000}\PY{p}{)}\PY{o}{.}\PY{n}{result}\PY{p}{(}\PY{p}{)}\PY{o}{.}\PY{n}{get\PYZus{}counts}\PY{p}{(}\PY{p}{)} \PY{c+c1}{\PYZsh{} we run the simulation and get the counts}
\PY{n}{plot\PYZus{}histogram}\PY{p}{(}\PY{n}{counts}\PY{p}{)} \PY{c+c1}{\PYZsh{} let us plot a histogram to see the possible outcomes and corresponding probabilities}
\end{Verbatim}
\end{tcolorbox}
 
            
\prompt{Out}{outcolor}{78}{}
    
    \begin{center}
    \adjustimage{max size={0.9\linewidth}{0.9\paperheight}}{output_23_0.png}
    \end{center}
    { \hspace*{\fill} \\}
    

    \begin{tcolorbox}[breakable, size=fbox, boxrule=1pt, pad at break*=1mm,colback=cellbackground, colframe=cellborder]
\prompt{In}{incolor}{79}{\boxspacing}
\begin{Verbatim}[commandchars=\\\{\}]
\PY{k+kn}{from} \PY{n+nn}{qc\PYZus{}grader}\PY{n+nn}{.}\PY{n+nn}{challenges}\PY{n+nn}{.}\PY{n+nn}{qgss\PYZus{}2022} \PY{k+kn}{import} \PY{n}{grade\PYZus{}lab1\PYZus{}ex6}

\PY{c+c1}{\PYZsh{} Note that the grading function is expecting a quantum circuit without measurements}
\PY{n}{grade\PYZus{}lab1\PYZus{}ex6}\PY{p}{(}\PY{n}{lab1\PYZus{}ex6}\PY{p}{(}\PY{p}{)}\PY{p}{)}
\end{Verbatim}
\end{tcolorbox}

    \begin{Verbatim}[commandchars=\\\{\}]
Submitting your answer. Please wait{\ldots}
Congratulations 🎉! Your answer is correct and has been submitted.
    \end{Verbatim}

    Congratulations for finishing these introductory exercises! Hopefully,
they got you more familiar with the Bloch sphere and basic quantum
gates.

    \hypertarget{part-ii-quantum-circuits-and-complexity}{%
\section{Part II: Quantum Circuits and
Complexity}\label{part-ii-quantum-circuits-and-complexity}}

\hypertarget{ii.1-complexity-of-the-number-of-gates}{%
\subsection{II.1: Complexity of the Number of
Gates}\label{ii.1-complexity-of-the-number-of-gates}}

We have seen different complexity classes and the big \(O\) notation and
how it can be used with Quantum Algorithms, when using gates. One
possible way to calculate the complexity of an algorithm is to just
count the number of gates used.

Another often seen measure is to count the number of multi qubit gates
rather than all gates, since they are normally ``more expensive'' than
other gates. In our case ``more expensive'' means that they often have a
way higher error rate (around 10 times higher) compared to single qubit
gates.

So, lets look again at the GHZ state and count the number of gates as
well as the number of multi qubit gates:

    \begin{tcolorbox}[breakable, size=fbox, boxrule=1pt, pad at break*=1mm,colback=cellbackground, colframe=cellborder]
\prompt{In}{incolor}{80}{\boxspacing}
\begin{Verbatim}[commandchars=\\\{\}]
\PY{n}{qc} \PY{o}{=} \PY{n}{QuantumCircuit}\PY{p}{(}\PY{l+m+mi}{4}\PY{p}{)}

\PY{n}{qc}\PY{o}{.}\PY{n}{h}\PY{p}{(}\PY{l+m+mi}{0}\PY{p}{)}
\PY{n}{qc}\PY{o}{.}\PY{n}{cx}\PY{p}{(}\PY{l+m+mi}{0}\PY{p}{,}\PY{l+m+mi}{1}\PY{p}{)}
\PY{n}{qc}\PY{o}{.}\PY{n}{cx}\PY{p}{(}\PY{l+m+mi}{0}\PY{p}{,}\PY{l+m+mi}{2}\PY{p}{)}

\PY{n+nb}{print}\PY{p}{(}\PY{n}{qc}\PY{o}{.}\PY{n}{size}\PY{p}{(}\PY{p}{)}\PY{p}{)}
\PY{n+nb}{print}\PY{p}{(}\PY{n}{qc}\PY{o}{.}\PY{n}{num\PYZus{}nonlocal\PYZus{}gates}\PY{p}{(}\PY{p}{)}\PY{p}{)}
\end{Verbatim}
\end{tcolorbox}

    \begin{Verbatim}[commandchars=\\\{\}]
3
2
    \end{Verbatim}

    Of course, in this example the number of gates is three and the number
of multi qubit gates is 2, this might be obvious in this case, since we
added gate by gate ourselves, but it will be less obvious when we use
algorithms to construct our quantum circuits.

    \hypertarget{ii.1.1-quantum-fourier-transform-example}{%
\subsubsection{II.1.1: Quantum Fourier Transform
Example}\label{ii.1.1-quantum-fourier-transform-example}}

Let's look at the example of the Quantum Fourier transform which was
also shown in the lecture. If you want to learn more about it or if you
want to refresh your knowledge you can read the following chapter:
https://qiskit.org/textbook/ch-algorithms/quantum-fourier-transform.html

    \begin{tcolorbox}[breakable, size=fbox, boxrule=1pt, pad at break*=1mm,colback=cellbackground, colframe=cellborder]
\prompt{In}{incolor}{81}{\boxspacing}
\begin{Verbatim}[commandchars=\\\{\}]
\PY{k}{def} \PY{n+nf}{qft\PYZus{}rotations}\PY{p}{(}\PY{n}{circuit}\PY{p}{,} \PY{n}{n}\PY{p}{)}\PY{p}{:}
    \PY{l+s+sd}{\PYZdq{}\PYZdq{}\PYZdq{}Performs qft on the first n qubits in circuit (without swaps)\PYZdq{}\PYZdq{}\PYZdq{}}
    \PY{k}{if} \PY{n}{n} \PY{o}{==} \PY{l+m+mi}{0}\PY{p}{:}
        \PY{k}{return} \PY{n}{circuit}
    \PY{n}{n} \PY{o}{\PYZhy{}}\PY{o}{=} \PY{l+m+mi}{1}
    \PY{n}{circuit}\PY{o}{.}\PY{n}{h}\PY{p}{(}\PY{n}{n}\PY{p}{)}
    \PY{k}{for} \PY{n}{qubit} \PY{o+ow}{in} \PY{n+nb}{range}\PY{p}{(}\PY{n}{n}\PY{p}{)}\PY{p}{:}
        \PY{n}{circuit}\PY{o}{.}\PY{n}{cp}\PY{p}{(}\PY{n}{pi}\PY{o}{/}\PY{l+m+mi}{2}\PY{o}{*}\PY{o}{*}\PY{p}{(}\PY{n}{n}\PY{o}{\PYZhy{}}\PY{n}{qubit}\PY{p}{)}\PY{p}{,} \PY{n}{qubit}\PY{p}{,} \PY{n}{n}\PY{p}{)}
    \PY{c+c1}{\PYZsh{} At the end of our function, we call the same function again on}
    \PY{c+c1}{\PYZsh{} the next qubits (we reduced n by one earlier in the function)}
    \PY{n}{qft\PYZus{}rotations}\PY{p}{(}\PY{n}{circuit}\PY{p}{,} \PY{n}{n}\PY{p}{)}
    
\PY{k}{def} \PY{n+nf}{swap\PYZus{}registers}\PY{p}{(}\PY{n}{circuit}\PY{p}{,} \PY{n}{n}\PY{p}{)}\PY{p}{:}
    \PY{l+s+sd}{\PYZdq{}\PYZdq{}\PYZdq{}Swaps registers to match the definition\PYZdq{}\PYZdq{}\PYZdq{}}
    \PY{k}{for} \PY{n}{qubit} \PY{o+ow}{in} \PY{n+nb}{range}\PY{p}{(}\PY{n}{n}\PY{o}{/}\PY{o}{/}\PY{l+m+mi}{2}\PY{p}{)}\PY{p}{:}
        \PY{n}{circuit}\PY{o}{.}\PY{n}{swap}\PY{p}{(}\PY{n}{qubit}\PY{p}{,} \PY{n}{n}\PY{o}{\PYZhy{}}\PY{n}{qubit}\PY{o}{\PYZhy{}}\PY{l+m+mi}{1}\PY{p}{)}
    \PY{k}{return} \PY{n}{circuit}

\PY{k}{def} \PY{n+nf}{qft}\PY{p}{(}\PY{n}{circuit}\PY{p}{,} \PY{n}{n}\PY{p}{)}\PY{p}{:}
    \PY{l+s+sd}{\PYZdq{}\PYZdq{}\PYZdq{}QFT on the first n qubits in circuit\PYZdq{}\PYZdq{}\PYZdq{}}
    \PY{n}{qft\PYZus{}rotations}\PY{p}{(}\PY{n}{circuit}\PY{p}{,} \PY{n}{n}\PY{p}{)}
    \PY{n}{swap\PYZus{}registers}\PY{p}{(}\PY{n}{circuit}\PY{p}{,} \PY{n}{n}\PY{p}{)}
    \PY{k}{return} \PY{n}{circuit}
\end{Verbatim}
\end{tcolorbox}

    For now, let's not do the whole Quantum Fourier Transformation but only
the rotations. And let's apply them to the quantum state we defined
above and measure the number of operations used:

    \begin{tcolorbox}[breakable, size=fbox, boxrule=1pt, pad at break*=1mm,colback=cellbackground, colframe=cellborder]
\prompt{In}{incolor}{82}{\boxspacing}
\begin{Verbatim}[commandchars=\\\{\}]
\PY{n}{n}\PY{o}{=}\PY{l+m+mi}{10}
\PY{n}{qc} \PY{o}{=} \PY{n}{QuantumCircuit}\PY{p}{(}\PY{n}{n}\PY{p}{)}
\PY{n}{qft\PYZus{}rotations}\PY{p}{(}\PY{n}{qc}\PY{p}{,}\PY{n}{n}\PY{p}{)}
\PY{n+nb}{print}\PY{p}{(}\PY{n}{qc}\PY{o}{.}\PY{n}{size}\PY{p}{(}\PY{p}{)}\PY{p}{)}
\PY{n+nb}{print}\PY{p}{(}\PY{n}{qc}\PY{o}{.}\PY{n}{num\PYZus{}nonlocal\PYZus{}gates}\PY{p}{(}\PY{p}{)}\PY{p}{)}
\PY{n}{qc}\PY{o}{.}\PY{n}{draw}\PY{p}{(}\PY{p}{)}
\end{Verbatim}
\end{tcolorbox}

    \begin{Verbatim}[commandchars=\\\{\}]
55
45
    \end{Verbatim}
 
            
\prompt{Out}{outcolor}{82}{}
    
    \begin{center}
    \adjustimage{max size={0.9\linewidth}{0.9\paperheight}}{output_32_1.png}
    \end{center}
    { \hspace*{\fill} \\}
    

    As we can see, the first 3 gates are from our GHZ state and the other 6
gates are coming from the Fourier transformation, and it looks the same
no matter on which state we apply it, it will always take the same
amount of gates. This means that we can now for the next examples just
consider the all 0 state (the base state which needs no gates to
construct) and just consider the number of gates of the Fourier
transformation itself.

In the textbook we can use the scalable circuit widget to see how the
circuit for the Fourier transformation gets bigger as we apply it to
more circuits. See here:
https://qiskit.org/textbook/ch-algorithms/quantum-fourier-transform.html\#8.2-General-QFT-Function-

We are, however, less interested in how the circuits for more qubits
look like, as in how many gates the circuit will need. We saw that for 3
qubits, the quantum Fourier transformation (without the swaps) needs 6
gates in total. How many does it need for 4, 5, 10, 100, 200 qubits?

NOTE: We do ask for the total number of gates not the number of multi
qubit gates. (Although this can be easily calculated as well, if you
have found the solution for the total number of gates).

    \begin{tcolorbox}[breakable, size=fbox, boxrule=1pt, pad at break*=1mm,colback=cellbackground, colframe=cellborder]
\prompt{In}{incolor}{83}{\boxspacing}
\begin{Verbatim}[commandchars=\\\{\}]
\PY{k}{def} \PY{n+nf}{lab1\PYZus{}ex7}\PY{p}{(}\PY{n}{n}\PY{p}{:}\PY{n+nb}{int}\PY{p}{)} \PY{o}{\PYZhy{}}\PY{o}{\PYZgt{}} \PY{n+nb}{int}\PY{p}{:}
    \PY{c+c1}{\PYZsh{}Here we want you to build a function calculating the number of gates needed by the fourier rotation for n qubits.}
    
    \PY{n}{numberOfGates}\PY{o}{=}\PY{l+m+mi}{0}
    \PY{n}{qc} \PY{o}{=} \PY{n}{QuantumCircuit}\PY{p}{(}\PY{n}{n}\PY{p}{)}
    \PY{n}{qft\PYZus{}rotations}\PY{p}{(}\PY{n}{qc}\PY{p}{,}\PY{n}{n}\PY{p}{)}
    \PY{k}{return} \PY{n}{qc}\PY{o}{.}\PY{n}{size}\PY{p}{(}\PY{p}{)}

\PY{n+nb}{print}\PY{p}{(}\PY{n}{lab1\PYZus{}ex7}\PY{p}{(}\PY{l+m+mi}{3}\PY{p}{)}\PY{p}{)}
\PY{n+nb}{print}\PY{p}{(}\PY{n}{lab1\PYZus{}ex7}\PY{p}{(}\PY{l+m+mi}{4}\PY{p}{)}\PY{p}{)}
\PY{n+nb}{print}\PY{p}{(}\PY{n}{lab1\PYZus{}ex7}\PY{p}{(}\PY{l+m+mi}{5}\PY{p}{)}\PY{p}{)}
\PY{n+nb}{print}\PY{p}{(}\PY{n}{lab1\PYZus{}ex7}\PY{p}{(}\PY{l+m+mi}{10}\PY{p}{)}\PY{p}{)}
\PY{n+nb}{print}\PY{p}{(}\PY{n}{lab1\PYZus{}ex7}\PY{p}{(}\PY{l+m+mi}{100}\PY{p}{)}\PY{p}{)}
\PY{n+nb}{print}\PY{p}{(}\PY{n}{lab1\PYZus{}ex7}\PY{p}{(}\PY{l+m+mi}{200}\PY{p}{)}\PY{p}{)}

\PY{c+c1}{\PYZsh{} \PYZsh{} d\PYZus{}\PYZus{}\PYZus{}\PYZus{}\PYZus{}\PYZus{}\PYZus{}\PYZus{}\PYZus{}\PYZus{}\PYZus{}\PYZus{}\PYZus{}\PYZus{}\PYZus{}\PYZus{}\PYZus{}\PYZus{}\PYZus{}\PYZus{}\PYZus{}\PYZus{}\PYZus{}\PYZhy{}      }
\PY{c+c1}{\PYZsh{} 1 3 6 10 15 21 28}
\PY{c+c1}{\PYZsh{} 1 2 3  4  5  6  7}
\end{Verbatim}
\end{tcolorbox}

    \begin{Verbatim}[commandchars=\\\{\}]
6
10
15
55
5050
20100
    \end{Verbatim}

    \begin{tcolorbox}[breakable, size=fbox, boxrule=1pt, pad at break*=1mm,colback=cellbackground, colframe=cellborder]
\prompt{In}{incolor}{84}{\boxspacing}
\begin{Verbatim}[commandchars=\\\{\}]
\PY{c+c1}{\PYZsh{} Lab 1, Exercise 7}
\PY{k+kn}{from} \PY{n+nn}{qc\PYZus{}grader}\PY{n+nn}{.}\PY{n+nn}{challenges}\PY{n+nn}{.}\PY{n+nn}{qgss\PYZus{}2022} \PY{k+kn}{import} \PY{n}{grade\PYZus{}lab1\PYZus{}ex7}

\PY{c+c1}{\PYZsh{} Note that the grading function is expecting as input a function! }
\PY{c+c1}{\PYZsh{}(And the function takes n as an input and outputs the number of gates constructed)}
\PY{n}{grade\PYZus{}lab1\PYZus{}ex7}\PY{p}{(}\PY{n}{lab1\PYZus{}ex7}\PY{p}{)}
\end{Verbatim}
\end{tcolorbox}

    \begin{Verbatim}[commandchars=\\\{\}]
Submitting your answer. Please wait{\ldots}
Congratulations 🎉! Your answer is correct and has been submitted.
    \end{Verbatim}

    As you have seen above, the algorithm for the Quantum Fourier Transform
needs \(O(n^2)\) gates and one can also easily see that it also needs
\(O(n^2)\) two qubit gates.

If the algorithm would not have been recursive, and instead just uses
several loops, this would have been even easier to see.

So, if you ever have problems analysing the complexity of a (recursive)
algorithm, try to rewrite it using simple loops.

    \hypertarget{ii.2-complexity-of-the-depth-of-a-circuit}{%
\subsection{II.2: Complexity of the Depth of a
Circuit}\label{ii.2-complexity-of-the-depth-of-a-circuit}}

When it comes to how well a circuit runs on an actual Quantum Computer
the number of gates is not the only important factor.

The depth of the circuit tells how many ``layers'' of quantum gates,
executed in parallel, it takes to complete the computation defined by
the circuit. More information about it can be found here, especially the
animation comparing it to Tetris can help to understand the concept of
depth (open ``Quantum Circuit Properties'' to see it):
https://qiskit.org/documentation/apidoc/circuit.html\#supplementary-information

Now we look at two simple examples to show what the depth of a circuit
is:

    \begin{tcolorbox}[breakable, size=fbox, boxrule=1pt, pad at break*=1mm,colback=cellbackground, colframe=cellborder]
\prompt{In}{incolor}{85}{\boxspacing}
\begin{Verbatim}[commandchars=\\\{\}]
\PY{n}{qc} \PY{o}{=} \PY{n}{QuantumCircuit}\PY{p}{(}\PY{l+m+mi}{4}\PY{p}{)}

\PY{n}{qc}\PY{o}{.}\PY{n}{h}\PY{p}{(}\PY{l+m+mi}{0}\PY{p}{)}
\PY{n}{qc}\PY{o}{.}\PY{n}{s}\PY{p}{(}\PY{l+m+mi}{0}\PY{p}{)}
\PY{n}{qc}\PY{o}{.}\PY{n}{s}\PY{p}{(}\PY{l+m+mi}{0}\PY{p}{)}
\PY{n}{qc}\PY{o}{.}\PY{n}{s}\PY{p}{(}\PY{l+m+mi}{0}\PY{p}{)}
\PY{n}{qc}\PY{o}{.}\PY{n}{cx}\PY{p}{(}\PY{l+m+mi}{0}\PY{p}{,}\PY{l+m+mi}{1}\PY{p}{)}
\PY{n}{qc}\PY{o}{.}\PY{n}{cx}\PY{p}{(}\PY{l+m+mi}{1}\PY{p}{,}\PY{l+m+mi}{3}\PY{p}{)}

\PY{n+nb}{print}\PY{p}{(}\PY{n}{qc}\PY{o}{.}\PY{n}{depth}\PY{p}{(}\PY{p}{)}\PY{p}{)}
\PY{n}{qc}\PY{o}{.}\PY{n}{draw}\PY{p}{(}\PY{p}{)}
\end{Verbatim}
\end{tcolorbox}

    \begin{Verbatim}[commandchars=\\\{\}]
6
    \end{Verbatim}
 
            
\prompt{Out}{outcolor}{85}{}
    
    \begin{center}
    \adjustimage{max size={0.9\linewidth}{0.9\paperheight}}{output_38_1.png}
    \end{center}
    { \hspace*{\fill} \\}
    

    \begin{tcolorbox}[breakable, size=fbox, boxrule=1pt, pad at break*=1mm,colback=cellbackground, colframe=cellborder]
\prompt{In}{incolor}{86}{\boxspacing}
\begin{Verbatim}[commandchars=\\\{\}]
\PY{n}{qc2} \PY{o}{=} \PY{n}{QuantumCircuit}\PY{p}{(}\PY{l+m+mi}{4}\PY{p}{)}

\PY{n}{qc2}\PY{o}{.}\PY{n}{h}\PY{p}{(}\PY{l+m+mi}{0}\PY{p}{)}
\PY{n}{qc2}\PY{o}{.}\PY{n}{s}\PY{p}{(}\PY{l+m+mi}{1}\PY{p}{)}


\PY{n}{qc2}\PY{o}{.}\PY{n}{cx}\PY{p}{(}\PY{l+m+mi}{0}\PY{p}{,}\PY{l+m+mi}{1}\PY{p}{)}

\PY{n}{qc2}\PY{o}{.}\PY{n}{s}\PY{p}{(}\PY{l+m+mi}{2}\PY{p}{)}
\PY{n}{qc2}\PY{o}{.}\PY{n}{s}\PY{p}{(}\PY{l+m+mi}{3}\PY{p}{)}

\PY{n}{qc2}\PY{o}{.}\PY{n}{cx}\PY{p}{(}\PY{l+m+mi}{2}\PY{p}{,}\PY{l+m+mi}{3}\PY{p}{)}

\PY{n+nb}{print}\PY{p}{(}\PY{n}{qc2}\PY{o}{.}\PY{n}{depth}\PY{p}{(}\PY{p}{)}\PY{p}{)}

\PY{n}{qc2}\PY{o}{.}\PY{n}{draw}\PY{p}{(}\PY{p}{)}
\end{Verbatim}
\end{tcolorbox}

    \begin{Verbatim}[commandchars=\\\{\}]
2
    \end{Verbatim}
 
            
\prompt{Out}{outcolor}{86}{}
    
    \begin{center}
    \adjustimage{max size={0.9\linewidth}{0.9\paperheight}}{output_39_1.png}
    \end{center}
    { \hspace*{\fill} \\}
    

    The length of the circuit above corresponds with the width. And as you
can see, both circuits have the same number of gates, but the first
circuit has a much higher depth, because all the gates depend on the
gates before, so nothing can be done in parallel.

In short, the more of the gates can be applied in parallel, because they
apply to different qubits, the lower will the depth of a circuit be. The
lower bound on the depth of a circuit (if it has only single qubit gates
and they are evenly distributed) is the number of gates divided by the
number of qubits.

On the other hand, if every gate in a quantum circuit depends on the
same qubit, the depth will be the same as the number of qubits.

    \hypertarget{ii.2.2-fully-entangled-state-example}{%
\subsection{II.2.2: Fully Entangled State
Example}\label{ii.2.2-fully-entangled-state-example}}

Let's take a look at the example of the naive implementation of a fully
entangled state:

    \begin{tcolorbox}[breakable, size=fbox, boxrule=1pt, pad at break*=1mm,colback=cellbackground, colframe=cellborder]
\prompt{In}{incolor}{87}{\boxspacing}
\begin{Verbatim}[commandchars=\\\{\}]
\PY{n}{qc} \PY{o}{=} \PY{n}{QuantumCircuit}\PY{p}{(}\PY{l+m+mi}{16}\PY{p}{)}

\PY{c+c1}{\PYZsh{}Step 1: Preparing the first qubit in superposition}
\PY{n}{qc}\PY{o}{.}\PY{n}{h}\PY{p}{(}\PY{l+m+mi}{0}\PY{p}{)}

\PY{c+c1}{\PYZsh{}Step 2: Entangling all other qubits with it (1 is included 16 is exclude)}
\PY{k}{for} \PY{n}{x} \PY{o+ow}{in} \PY{n+nb}{range}\PY{p}{(}\PY{l+m+mi}{1}\PY{p}{,} \PY{l+m+mi}{16}\PY{p}{)}\PY{p}{:}
  \PY{n}{qc}\PY{o}{.}\PY{n}{cx}\PY{p}{(}\PY{l+m+mi}{0}\PY{p}{,}\PY{n}{x}\PY{p}{)}

\PY{n+nb}{print}\PY{p}{(}\PY{n}{qc}\PY{o}{.}\PY{n}{depth}\PY{p}{(}\PY{p}{)}\PY{p}{)}
\PY{n}{qc}\PY{o}{.}\PY{n}{draw}\PY{p}{(}\PY{p}{)}
\end{Verbatim}
\end{tcolorbox}

    \begin{Verbatim}[commandchars=\\\{\}]
16
    \end{Verbatim}
 
            
\prompt{Out}{outcolor}{87}{}
    
    \begin{center}
    \adjustimage{max size={0.9\linewidth}{0.9\paperheight}}{output_42_1.png}
    \end{center}
    { \hspace*{\fill} \\}
    

    As we can see the above quantum circuit has its depth equal to its
number of gates. Step 1 adds a depth of 1 and step 2 adds a depth of 15.

Let's try to do this better! Its quite clear that we can't do Step 1
better, but step 2 can be done a lot better. So lets try to find a
solution, which only uses a depth of 4, instead of 15!

Hint: Lets think about what kind of asymptotic running time would cause
only 4 operations. And don't forget that the final depth will be 5 (Step
1 and 2 combined).

    \begin{tcolorbox}[breakable, size=fbox, boxrule=1pt, pad at break*=1mm,colback=cellbackground, colframe=cellborder]
\prompt{In}{incolor}{88}{\boxspacing}
\begin{Verbatim}[commandchars=\\\{\}]
\PY{k}{def} \PY{n+nf}{lab1\PYZus{}ex8}\PY{p}{(}\PY{p}{)}\PY{p}{:}
    \PY{n}{qc}\PY{o}{=}\PY{n}{QuantumCircuit}\PY{p}{(}\PY{l+m+mi}{16}\PY{p}{)}
    \PY{n}{qc}\PY{o}{.}\PY{n}{h}\PY{p}{(}\PY{l+m+mi}{0}\PY{p}{)}
    
    \PY{n}{qc}\PY{o}{.}\PY{n}{cx}\PY{p}{(}\PY{l+m+mi}{0}\PY{p}{,}\PY{l+m+mi}{1}\PY{p}{)}
    
    \PY{n}{qc}\PY{o}{.}\PY{n}{cx}\PY{p}{(}\PY{l+m+mi}{0}\PY{p}{,}\PY{l+m+mi}{2}\PY{p}{)}
    \PY{n}{qc}\PY{o}{.}\PY{n}{cx}\PY{p}{(}\PY{l+m+mi}{1}\PY{p}{,}\PY{l+m+mi}{3}\PY{p}{)}
    
    \PY{n}{qc}\PY{o}{.}\PY{n}{cx}\PY{p}{(}\PY{l+m+mi}{0}\PY{p}{,}\PY{l+m+mi}{4}\PY{p}{)}
    \PY{n}{qc}\PY{o}{.}\PY{n}{cx}\PY{p}{(}\PY{l+m+mi}{1}\PY{p}{,}\PY{l+m+mi}{5}\PY{p}{)}
    \PY{n}{qc}\PY{o}{.}\PY{n}{cx}\PY{p}{(}\PY{l+m+mi}{2}\PY{p}{,}\PY{l+m+mi}{6}\PY{p}{)}
    \PY{n}{qc}\PY{o}{.}\PY{n}{cx}\PY{p}{(}\PY{l+m+mi}{3}\PY{p}{,}\PY{l+m+mi}{7}\PY{p}{)}
    
    \PY{n}{qc}\PY{o}{.}\PY{n}{cx}\PY{p}{(}\PY{l+m+mi}{0}\PY{p}{,}\PY{l+m+mi}{8}\PY{p}{)}
    \PY{n}{qc}\PY{o}{.}\PY{n}{cx}\PY{p}{(}\PY{l+m+mi}{1}\PY{p}{,}\PY{l+m+mi}{9}\PY{p}{)}
    \PY{n}{qc}\PY{o}{.}\PY{n}{cx}\PY{p}{(}\PY{l+m+mi}{2}\PY{p}{,}\PY{l+m+mi}{10}\PY{p}{)}
    \PY{n}{qc}\PY{o}{.}\PY{n}{cx}\PY{p}{(}\PY{l+m+mi}{3}\PY{p}{,}\PY{l+m+mi}{11}\PY{p}{)}
    \PY{n}{qc}\PY{o}{.}\PY{n}{cx}\PY{p}{(}\PY{l+m+mi}{4}\PY{p}{,}\PY{l+m+mi}{12}\PY{p}{)}
    \PY{n}{qc}\PY{o}{.}\PY{n}{cx}\PY{p}{(}\PY{l+m+mi}{5}\PY{p}{,}\PY{l+m+mi}{13}\PY{p}{)}
    \PY{n}{qc}\PY{o}{.}\PY{n}{cx}\PY{p}{(}\PY{l+m+mi}{6}\PY{p}{,}\PY{l+m+mi}{14}\PY{p}{)}
    \PY{n}{qc}\PY{o}{.}\PY{n}{cx}\PY{p}{(}\PY{l+m+mi}{7}\PY{p}{,}\PY{l+m+mi}{15}\PY{p}{)}



    
    

    
    \PY{c+c1}{\PYZsh{}}
    \PY{c+c1}{\PYZsh{}}
    \PY{c+c1}{\PYZsh{} FILL YOUR CODE IN HERE}
    \PY{c+c1}{\PYZsh{}}
    \PY{k}{return} \PY{n}{qc}

\PY{n}{qc} \PY{o}{=} \PY{n}{lab1\PYZus{}ex8}\PY{p}{(}\PY{p}{)}
\PY{n+nb}{print}\PY{p}{(}\PY{n}{qc}\PY{o}{.}\PY{n}{depth}\PY{p}{(}\PY{p}{)}\PY{p}{)}
\PY{n}{qc}\PY{o}{.}\PY{n}{draw}\PY{p}{(}\PY{p}{)}
\end{Verbatim}
\end{tcolorbox}

    \begin{Verbatim}[commandchars=\\\{\}]
5
    \end{Verbatim}
 
            
\prompt{Out}{outcolor}{88}{}
    
    \begin{center}
    \adjustimage{max size={0.9\linewidth}{0.9\paperheight}}{output_44_1.png}
    \end{center}
    { \hspace*{\fill} \\}
    

    \begin{tcolorbox}[breakable, size=fbox, boxrule=1pt, pad at break*=1mm,colback=cellbackground, colframe=cellborder]
\prompt{In}{incolor}{89}{\boxspacing}
\begin{Verbatim}[commandchars=\\\{\}]
\PY{k+kn}{from} \PY{n+nn}{qc\PYZus{}grader}\PY{n+nn}{.}\PY{n+nn}{challenges}\PY{n+nn}{.}\PY{n+nn}{qgss\PYZus{}2022} \PY{k+kn}{import} \PY{n}{grade\PYZus{}lab1\PYZus{}ex8}
\PY{c+c1}{\PYZsh{} Note that the grading function is expecting a quantum circuit without measurements}
\PY{n}{grade\PYZus{}lab1\PYZus{}ex8}\PY{p}{(}\PY{n}{lab1\PYZus{}ex8}\PY{p}{(}\PY{p}{)}\PY{p}{)}
\end{Verbatim}
\end{tcolorbox}

    \begin{Verbatim}[commandchars=\\\{\}]
Submitting your answer. Please wait{\ldots}
Congratulations 🎉! Your answer is correct and has been submitted.
    \end{Verbatim}

    Congratulation! You just improved the depth of a circuit by a factor of
9 thanks to your understanding of asymptotic complexity and quantum
circuits.


    % Add a bibliography block to the postdoc
    
    
    
\end{document}
